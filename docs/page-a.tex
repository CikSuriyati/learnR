% Options for packages loaded elsewhere
\PassOptionsToPackage{unicode}{hyperref}
\PassOptionsToPackage{hyphens}{url}
%
\documentclass[
]{article}
\usepackage{amsmath,amssymb}
\usepackage{lmodern}
\usepackage{iftex}
\ifPDFTeX
  \usepackage[T1]{fontenc}
  \usepackage[utf8]{inputenc}
  \usepackage{textcomp} % provide euro and other symbols
\else % if luatex or xetex
  \usepackage{unicode-math}
  \defaultfontfeatures{Scale=MatchLowercase}
  \defaultfontfeatures[\rmfamily]{Ligatures=TeX,Scale=1}
\fi
% Use upquote if available, for straight quotes in verbatim environments
\IfFileExists{upquote.sty}{\usepackage{upquote}}{}
\IfFileExists{microtype.sty}{% use microtype if available
  \usepackage[]{microtype}
  \UseMicrotypeSet[protrusion]{basicmath} % disable protrusion for tt fonts
}{}
\makeatletter
\@ifundefined{KOMAClassName}{% if non-KOMA class
  \IfFileExists{parskip.sty}{%
    \usepackage{parskip}
  }{% else
    \setlength{\parindent}{0pt}
    \setlength{\parskip}{6pt plus 2pt minus 1pt}}
}{% if KOMA class
  \KOMAoptions{parskip=half}}
\makeatother
\usepackage{xcolor}
\usepackage[margin=1in]{geometry}
\usepackage{color}
\usepackage{fancyvrb}
\newcommand{\VerbBar}{|}
\newcommand{\VERB}{\Verb[commandchars=\\\{\}]}
\DefineVerbatimEnvironment{Highlighting}{Verbatim}{commandchars=\\\{\}}
% Add ',fontsize=\small' for more characters per line
\usepackage{framed}
\definecolor{shadecolor}{RGB}{248,248,248}
\newenvironment{Shaded}{\begin{snugshade}}{\end{snugshade}}
\newcommand{\AlertTok}[1]{\textcolor[rgb]{0.94,0.16,0.16}{#1}}
\newcommand{\AnnotationTok}[1]{\textcolor[rgb]{0.56,0.35,0.01}{\textbf{\textit{#1}}}}
\newcommand{\AttributeTok}[1]{\textcolor[rgb]{0.77,0.63,0.00}{#1}}
\newcommand{\BaseNTok}[1]{\textcolor[rgb]{0.00,0.00,0.81}{#1}}
\newcommand{\BuiltInTok}[1]{#1}
\newcommand{\CharTok}[1]{\textcolor[rgb]{0.31,0.60,0.02}{#1}}
\newcommand{\CommentTok}[1]{\textcolor[rgb]{0.56,0.35,0.01}{\textit{#1}}}
\newcommand{\CommentVarTok}[1]{\textcolor[rgb]{0.56,0.35,0.01}{\textbf{\textit{#1}}}}
\newcommand{\ConstantTok}[1]{\textcolor[rgb]{0.00,0.00,0.00}{#1}}
\newcommand{\ControlFlowTok}[1]{\textcolor[rgb]{0.13,0.29,0.53}{\textbf{#1}}}
\newcommand{\DataTypeTok}[1]{\textcolor[rgb]{0.13,0.29,0.53}{#1}}
\newcommand{\DecValTok}[1]{\textcolor[rgb]{0.00,0.00,0.81}{#1}}
\newcommand{\DocumentationTok}[1]{\textcolor[rgb]{0.56,0.35,0.01}{\textbf{\textit{#1}}}}
\newcommand{\ErrorTok}[1]{\textcolor[rgb]{0.64,0.00,0.00}{\textbf{#1}}}
\newcommand{\ExtensionTok}[1]{#1}
\newcommand{\FloatTok}[1]{\textcolor[rgb]{0.00,0.00,0.81}{#1}}
\newcommand{\FunctionTok}[1]{\textcolor[rgb]{0.00,0.00,0.00}{#1}}
\newcommand{\ImportTok}[1]{#1}
\newcommand{\InformationTok}[1]{\textcolor[rgb]{0.56,0.35,0.01}{\textbf{\textit{#1}}}}
\newcommand{\KeywordTok}[1]{\textcolor[rgb]{0.13,0.29,0.53}{\textbf{#1}}}
\newcommand{\NormalTok}[1]{#1}
\newcommand{\OperatorTok}[1]{\textcolor[rgb]{0.81,0.36,0.00}{\textbf{#1}}}
\newcommand{\OtherTok}[1]{\textcolor[rgb]{0.56,0.35,0.01}{#1}}
\newcommand{\PreprocessorTok}[1]{\textcolor[rgb]{0.56,0.35,0.01}{\textit{#1}}}
\newcommand{\RegionMarkerTok}[1]{#1}
\newcommand{\SpecialCharTok}[1]{\textcolor[rgb]{0.00,0.00,0.00}{#1}}
\newcommand{\SpecialStringTok}[1]{\textcolor[rgb]{0.31,0.60,0.02}{#1}}
\newcommand{\StringTok}[1]{\textcolor[rgb]{0.31,0.60,0.02}{#1}}
\newcommand{\VariableTok}[1]{\textcolor[rgb]{0.00,0.00,0.00}{#1}}
\newcommand{\VerbatimStringTok}[1]{\textcolor[rgb]{0.31,0.60,0.02}{#1}}
\newcommand{\WarningTok}[1]{\textcolor[rgb]{0.56,0.35,0.01}{\textbf{\textit{#1}}}}
\usepackage{graphicx}
\makeatletter
\def\maxwidth{\ifdim\Gin@nat@width>\linewidth\linewidth\else\Gin@nat@width\fi}
\def\maxheight{\ifdim\Gin@nat@height>\textheight\textheight\else\Gin@nat@height\fi}
\makeatother
% Scale images if necessary, so that they will not overflow the page
% margins by default, and it is still possible to overwrite the defaults
% using explicit options in \includegraphics[width, height, ...]{}
\setkeys{Gin}{width=\maxwidth,height=\maxheight,keepaspectratio}
% Set default figure placement to htbp
\makeatletter
\def\fps@figure{htbp}
\makeatother
\setlength{\emergencystretch}{3em} % prevent overfull lines
\providecommand{\tightlist}{%
  \setlength{\itemsep}{0pt}\setlength{\parskip}{0pt}}
\setcounter{secnumdepth}{-\maxdimen} % remove section numbering
\usepackage{booktabs}
\usepackage{longtable}
\usepackage{array}
\usepackage{multirow}
\usepackage{wrapfig}
\usepackage{float}
\usepackage{colortbl}
\usepackage{pdflscape}
\usepackage{tabu}
\usepackage{threeparttable}
\usepackage{threeparttablex}
\usepackage[normalem]{ulem}
\usepackage{makecell}
\usepackage{xcolor}
\ifLuaTeX
  \usepackage{selnolig}  % disable illegal ligatures
\fi
\IfFileExists{bookmark.sty}{\usepackage{bookmark}}{\usepackage{hyperref}}
\IfFileExists{xurl.sty}{\usepackage{xurl}}{} % add URL line breaks if available
\urlstyle{same} % disable monospaced font for URLs
\hypersetup{
  pdftitle={Topic 1: Introduction to R language},
  hidelinks,
  pdfcreator={LaTeX via pandoc}}

\title{Topic 1: Introduction to R language}
\author{}
\date{\vspace{-2.5em}}

\begin{document}
\maketitle

In this topic in week 1, you will learn about :

\begin{itemize}
\tightlist
\item
  basic features of R programming
\item
  Logical vectors and relational operators
\end{itemize}

\hypertarget{section}{%
\section{}\label{section}}

\hypertarget{basic-features}{%
\subsection{Basic Features}\label{basic-features}}

\href{https://ciksuriyati.shinyapps.io/penguins/?_ga=2.111118085.1242787491.1673712812-716303559.1673712812}{EXERCISES}

\hypertarget{introduction-to-basic-features-of-r-programming}{%
\subsubsection{Introduction to basic features of R
programming}\label{introduction-to-basic-features-of-r-programming}}

\textbf{Basic Features of R Programming}

R is a powerful and popular programming language widely used for data
analysis, statistical computing, and graphical representation. It has
several essential features that make it a favorite choice among data
scientists, statisticians, and researchers. Below are some of the key
features of R programming:

\begin{enumerate}
\def\labelenumi{\arabic{enumi}.}
\tightlist
\item
  \textbf{Open Source}: R is an open-source language, which means it is
  freely available for anyone to use, modify, and distribute. The
  open-source nature encourages collaboration and continuous improvement
  of the language.
\item
  \textbf{Data Handling}: R provides efficient data handling
  capabilities, making it easy to import, manipulate, and transform
  data. It supports various data structures, such as vectors, matrices,
  data frames, and lists.
\item
  \textbf{Statistical Analysis}: R is primarily known for its extensive
  statistical capabilities. It offers a vast collection of built-in
  functions and packages for conducting various statistical analyses,
  including regression, hypothesis testing, ANOVA, and time series
  analysis.
\item
  \textbf{Graphics and Visualization}: R excels in producing
  high-quality visualizations and graphs. It offers powerful plotting
  functions and packages like ggplot2, lattice, and base graphics,
  enabling users to create insightful and publication-quality plots.
\item
  \textbf{Extensible}: R is an extensible language, allowing users to
  create their functions and packages to extend its functionality. The
  Comprehensive R Archive Network (CRAN) hosts thousands of
  user-contributed packages for various purposes.
\item
  \textbf{Reproducibility}: R promotes reproducibility in data analysis
  and research. Scripts written in R can be easily shared and rerun on
  different datasets, ensuring that others can replicate and verify the
  results. Documentation and Community Support: R has comprehensive
  documentation available through manuals, guides, and tutorials.
  Additionally, it has a vibrant and active community of users and
  developers who provide support through forums and mailing lists.
\item
  \textbf{Interactive Environment}: R offers an interactive environment
  with an interactive console and script editor, allowing users to
  execute commands interactively and see immediate results.
\item
  \textbf{Integration with Other Tools}: R can be integrated with other
  programming languages like Python, C++, and Java, enabling users to
  leverage their existing code and tools seamlessly.
\end{enumerate}

\textbf{Summary:} R programming offers a wide range of features for data
analysis, statistics, visualization, and reproducibility. Its
open-source nature, extensive statistical libraries, and active
community support make it a preferred choice for data analysis and
research in various domains. Whether you are a beginner or an
experienced data analyst, R provides a powerful and flexible environment
to explore, analyze, and visualize data effectively.

\hypertarget{step-by-step-guide-to-install-and-use-r}{%
\subsection{Step-by-Step Guide to Install and Use
R}\label{step-by-step-guide-to-install-and-use-r}}

Installing and using R is a straightforward process. Here's a
step-by-step guide to help you get started:

\begin{enumerate}
\def\labelenumi{\arabic{enumi}.}
\tightlist
\item
  \textbf{Download R}
\end{enumerate}

Go to the R Project website: \textbf{\url{https://www.r-project.org/}}
Click on ``CRAN'' (Comprehensive R Archive Network) under ``Download and
Install R.'' Select a CRAN mirror location near you. Choose the
appropriate operating system (Windows, macOS, or Linux). Download the R
installer for your operating system.

\begin{enumerate}
\def\labelenumi{\arabic{enumi}.}
\setcounter{enumi}{1}
\tightlist
\item
  \textbf{Install R}
\end{enumerate}

For \textbf{Windows}: Double-click the downloaded installer file and
follow the on-screen instructions to install R. For \textbf{macOS}:
Double-click the downloaded package file, and the installation process
will begin.

\textbf{Step 3: Launch R}

For \textbf{Windows}: You can find the R program in the Start Menu or on
your desktop. Double-click the R icon to launch the R console. For
\textbf{macOS}: You can find the R application in the Applications
folder. Double-click the R icon to launch the R console.

\textbf{Step 4: Using R Console}

After launching R, you will see the R console where you can type
commands and interact with R. To execute a command, type it in the
console and press Enter. For example, try typing \textbf{`print(``Hello,
R!'')'} and press Enter. R will print the message ``Hello, R!''.

\textbf{Step 5: Installing R Packages}

R has a vast collection of packages contributed by the community. To
install packages, you can use the \textbf{`install.packages()
function'}. For example, to install the \textbf{`ggplot2'} package for
data visualization, type \textbf{`install.packages(``ggplot2'')'} and
press Enter.

\textbf{Step 6: Loading and Using Packages}

Once a package is installed, you need to load it into your R session
before using its functions. To load a package, use the \textbf{library(
)} or \textbf{require( )} function. For example, to load the ggplot2
package, type \textbf{library(ggplot2)} and press Enter.

\textbf{Step 7: Using R Scripts}

Writing code directly in the console is suitable for small tasks, but
for more complex analyses or projects, it's best to use R scripts.
Create a new plain text file with a \textbf{.R} extension, e.g.,
\textbf{myscript.R}. Write your R code in the script file using a text
editor or an integrated development environment (IDE). Save the file and
then open it in the R console using the \textbf{source( )} function. For
example, \textbf{source(``myscript.R'')}.

\textbf{Step 8: Interacting with Plots}

When you create plots in R, they will typically appear in a separate
window or plot pane. You can interact with the plots using buttons or
options in the plot pane (for GUI-based interfaces) or by typing
additional commands to modify or save the plots.

\textbf{Step 9: Exiting R}

To exit R, type \textbf{q( )} or \textbf{quit( )} in the R console and
press Enter. Congratulations! You have successfully installed and
started using R. Now, you can explore R's vast capabilities for data
analysis, visualization, and statistical modeling to suit your needs.

\hypertarget{operators-in-r}{%
\subsection{Operators in R}\label{operators-in-r}}

\hypertarget{operators-in-r-1}{%
\subsubsection{Operators in R}\label{operators-in-r-1}}

Type of operators in R

\begin{itemize}
\tightlist
\item
  Assignment operators
\item
  Arithmetic operators
\item
  Relational operators
\item
  Logical operators
\end{itemize}

\textbf{\emph{1. ASSIGNMENT OPERATORS}}

The assignment operators in R allows you to assign data to a named
object in order to store the data.

\begin{verbatim}
##    Operator                                               Description
## 1:       <-                                      Leftwards assignment
## 2:        = Left assignment (not recommended) and argument assignment
## 3:      <<-        Left lexicographic assignment (for advanced users)
## 4:       ->                                     Rightwards assignment
## 5:      ->>       Right lexicographic assignment (for advanced users)
\end{verbatim}

Example

\begin{Shaded}
\begin{Highlighting}[]
\NormalTok{x }\OtherTok{\textless{}{-}} \DecValTok{2}
\NormalTok{y }\OtherTok{=} \DecValTok{6}
\DecValTok{3} \OtherTok{{-}\textgreater{}}\NormalTok{ z}
\NormalTok{w }\OtherTok{\textless{}\textless{}{-}} \DecValTok{7}
\DecValTok{8} \OtherTok{{-}\textgreater{}\textgreater{}}\NormalTok{ s}

\NormalTok{x}
\end{Highlighting}
\end{Shaded}

\begin{verbatim}
## [1] 2
\end{verbatim}

\begin{Shaded}
\begin{Highlighting}[]
\NormalTok{y}
\end{Highlighting}
\end{Shaded}

\begin{verbatim}
## [1] 6
\end{verbatim}

\begin{Shaded}
\begin{Highlighting}[]
\NormalTok{z}
\end{Highlighting}
\end{Shaded}

\begin{verbatim}
## [1] 3
\end{verbatim}

\begin{Shaded}
\begin{Highlighting}[]
\NormalTok{w}
\end{Highlighting}
\end{Shaded}

\begin{verbatim}
## [1] 7
\end{verbatim}

\begin{Shaded}
\begin{Highlighting}[]
\NormalTok{s}
\end{Highlighting}
\end{Shaded}

\begin{verbatim}
## [1] 8
\end{verbatim}

\textbf{\emph{2. ARITHMETIC OPERATORS}}

These operators are used to carry out mathematical operations like
addition and multiplication. Here is a list of arithmetic operators
available in R.

\begin{verbatim}
##   Operator                       Description
## 1        +                          Addition
## 2        –                       Subtraction
## 3        *                    Multiplication
## 4        /                          Division
## 5        ^                          Exponent
## 6       %% Modulus (Remainder from division)
## 7      %/%                  Integer Division
\end{verbatim}

Example

\begin{Shaded}
\begin{Highlighting}[]
\DecValTok{3} \SpecialCharTok{+} \DecValTok{5}   
\end{Highlighting}
\end{Shaded}

\begin{verbatim}
## [1] 8
\end{verbatim}

\begin{Shaded}
\begin{Highlighting}[]
\DecValTok{8} \SpecialCharTok{{-}} \DecValTok{3}   
\end{Highlighting}
\end{Shaded}

\begin{verbatim}
## [1] 5
\end{verbatim}

\begin{Shaded}
\begin{Highlighting}[]
\DecValTok{7} \SpecialCharTok{*} \DecValTok{5}   
\end{Highlighting}
\end{Shaded}

\begin{verbatim}
## [1] 35
\end{verbatim}

\begin{Shaded}
\begin{Highlighting}[]
\DecValTok{1}\SpecialCharTok{/}\DecValTok{2}     
\end{Highlighting}
\end{Shaded}

\begin{verbatim}
## [1] 0.5
\end{verbatim}

\begin{Shaded}
\begin{Highlighting}[]
\DecValTok{4} \SpecialCharTok{\^{}} \DecValTok{4}   
\end{Highlighting}
\end{Shaded}

\begin{verbatim}
## [1] 256
\end{verbatim}

\begin{Shaded}
\begin{Highlighting}[]
\DecValTok{4} \SpecialCharTok{**} \DecValTok{4}  
\end{Highlighting}
\end{Shaded}

\begin{verbatim}
## [1] 256
\end{verbatim}

\begin{Shaded}
\begin{Highlighting}[]
\DecValTok{5} \SpecialCharTok{\%\%} \DecValTok{3}  
\end{Highlighting}
\end{Shaded}

\begin{verbatim}
## [1] 2
\end{verbatim}

\begin{Shaded}
\begin{Highlighting}[]
\DecValTok{5} \SpecialCharTok{\%/\%} \DecValTok{3}
\end{Highlighting}
\end{Shaded}

\begin{verbatim}
## [1] 1
\end{verbatim}

\textbf{\emph{3. RELATIONAL/COMPARISON OPERATORS}}

Comparison or relational operators are designed to compare objects and
the output of these comparisons are of type boolean. To clarify, the
following table summarizes the R relational operators.

\begin{verbatim}
##    Operator              Description
## 1:        <                Less than
## 2:        >             Greater than
## 3:       <=    Less than or equal to
## 4:       >= Greater than or equal to
## 5:       ==                 Equal to
## 6:       !=             Not equal to
\end{verbatim}

Example

\begin{Shaded}
\begin{Highlighting}[]
\NormalTok{x }\OtherTok{\textless{}{-}} \DecValTok{3}
\NormalTok{y }\OtherTok{\textless{}{-}} \DecValTok{15}

\NormalTok{x}\SpecialCharTok{\textless{}}\NormalTok{y}
\end{Highlighting}
\end{Shaded}

\begin{verbatim}
## [1] TRUE
\end{verbatim}

\begin{Shaded}
\begin{Highlighting}[]
\NormalTok{x}\SpecialCharTok{\textgreater{}}\NormalTok{y}
\end{Highlighting}
\end{Shaded}

\begin{verbatim}
## [1] FALSE
\end{verbatim}

\begin{Shaded}
\begin{Highlighting}[]
\NormalTok{x}\SpecialCharTok{\textless{}=}\DecValTok{5}
\end{Highlighting}
\end{Shaded}

\begin{verbatim}
## [1] TRUE
\end{verbatim}

\begin{Shaded}
\begin{Highlighting}[]
\NormalTok{y}\SpecialCharTok{\textgreater{}=}\DecValTok{20}
\end{Highlighting}
\end{Shaded}

\begin{verbatim}
## [1] FALSE
\end{verbatim}

\begin{Shaded}
\begin{Highlighting}[]
\NormalTok{y }\SpecialCharTok{==} \DecValTok{16}
\end{Highlighting}
\end{Shaded}

\begin{verbatim}
## [1] FALSE
\end{verbatim}

\begin{Shaded}
\begin{Highlighting}[]
\NormalTok{x }\SpecialCharTok{!=} \DecValTok{5}
\end{Highlighting}
\end{Shaded}

\begin{verbatim}
## [1] TRUE
\end{verbatim}

\textbf{\emph{4. LOGICAL/BOOLEAN OPERATORS}}

\begin{verbatim}
##    Operator              Description
## 1:        !              Logical NOT
## 2:        & Element-wise logical AND
## 3:       &&              Logical AND
## 4:        |  Element-wise logical OR
## 5:       ||               Logical OR
\end{verbatim}

\emph{Example}

\begin{Shaded}
\begin{Highlighting}[]
\NormalTok{x}\OtherTok{\textless{}{-}}\FunctionTok{c}\NormalTok{(}\ConstantTok{TRUE}\NormalTok{, }\ConstantTok{FALSE}\NormalTok{, }\DecValTok{0}\NormalTok{,}\DecValTok{6}\NormalTok{)}
\NormalTok{x}
\end{Highlighting}
\end{Shaded}

\begin{verbatim}
## [1] 1 0 0 6
\end{verbatim}

\begin{Shaded}
\begin{Highlighting}[]
\NormalTok{x}\OtherTok{\textless{}{-}}\FunctionTok{c}\NormalTok{(}\ConstantTok{TRUE}\NormalTok{, }\ConstantTok{FALSE}\NormalTok{, }\DecValTok{0}\NormalTok{,}\DecValTok{6}\NormalTok{)}
\NormalTok{y }\OtherTok{\textless{}{-}} \FunctionTok{c}\NormalTok{(}\ConstantTok{FALSE}\NormalTok{,}\ConstantTok{TRUE}\NormalTok{,}\ConstantTok{FALSE}\NormalTok{,}\ConstantTok{TRUE}\NormalTok{)}

\SpecialCharTok{!}\NormalTok{x}
\end{Highlighting}
\end{Shaded}

\begin{verbatim}
## [1] FALSE  TRUE  TRUE FALSE
\end{verbatim}

Operators \& and \textbar{} perform element-wise operation producing
result having length of the longer operand.

\begin{Shaded}
\begin{Highlighting}[]
\NormalTok{x}\OtherTok{\textless{}{-}}\FunctionTok{c}\NormalTok{(}\ConstantTok{TRUE}\NormalTok{, }\ConstantTok{FALSE}\NormalTok{, }\DecValTok{0}\NormalTok{,}\DecValTok{6}\NormalTok{)}
\NormalTok{y }\OtherTok{\textless{}{-}} \FunctionTok{c}\NormalTok{(}\ConstantTok{FALSE}\NormalTok{,}\ConstantTok{TRUE}\NormalTok{,}\ConstantTok{FALSE}\NormalTok{,}\ConstantTok{TRUE}\NormalTok{)}

\NormalTok{x}\SpecialCharTok{\&}\NormalTok{y}
\end{Highlighting}
\end{Shaded}

\begin{verbatim}
## [1] FALSE FALSE FALSE  TRUE
\end{verbatim}

\begin{Shaded}
\begin{Highlighting}[]
\NormalTok{x}\OtherTok{\textless{}{-}}\FunctionTok{c}\NormalTok{(}\ConstantTok{TRUE}\NormalTok{, }\ConstantTok{FALSE}\NormalTok{, }\DecValTok{0}\NormalTok{,}\DecValTok{6}\NormalTok{)}
\NormalTok{y }\OtherTok{\textless{}{-}} \FunctionTok{c}\NormalTok{(}\ConstantTok{FALSE}\NormalTok{,}\ConstantTok{TRUE}\NormalTok{,}\ConstantTok{FALSE}\NormalTok{,}\ConstantTok{TRUE}\NormalTok{)}

\NormalTok{x}\SpecialCharTok{|}\NormalTok{y}
\end{Highlighting}
\end{Shaded}

\begin{verbatim}
## [1]  TRUE  TRUE FALSE  TRUE
\end{verbatim}

But \&\& and \textbar\textbar{} examines only the first element of the
operands resulting into a single length logical vector.

\begin{Shaded}
\begin{Highlighting}[]
\NormalTok{x}\OtherTok{\textless{}{-}}\FunctionTok{c}\NormalTok{(}\ConstantTok{TRUE}\NormalTok{, }\ConstantTok{FALSE}\NormalTok{, }\DecValTok{0}\NormalTok{,}\DecValTok{6}\NormalTok{)}
\NormalTok{y }\OtherTok{\textless{}{-}} \FunctionTok{c}\NormalTok{(}\ConstantTok{FALSE}\NormalTok{,}\ConstantTok{TRUE}\NormalTok{,}\ConstantTok{FALSE}\NormalTok{,}\ConstantTok{TRUE}\NormalTok{)}

\NormalTok{x}\SpecialCharTok{\&\&}\NormalTok{y}
\end{Highlighting}
\end{Shaded}

\begin{verbatim}
## [1] FALSE
\end{verbatim}

Zero is considered FALSE and non-zero numbers are taken as TRUE.

\begin{Shaded}
\begin{Highlighting}[]
\NormalTok{x}\OtherTok{\textless{}{-}}\FunctionTok{c}\NormalTok{(}\ConstantTok{TRUE}\NormalTok{, }\ConstantTok{FALSE}\NormalTok{, }\DecValTok{0}\NormalTok{,}\DecValTok{6}\NormalTok{)}
\NormalTok{y }\OtherTok{\textless{}{-}} \FunctionTok{c}\NormalTok{(}\ConstantTok{FALSE}\NormalTok{,}\ConstantTok{TRUE}\NormalTok{,}\ConstantTok{FALSE}\NormalTok{,}\ConstantTok{TRUE}\NormalTok{)}

\NormalTok{x}\SpecialCharTok{||}\NormalTok{y}
\end{Highlighting}
\end{Shaded}

\begin{verbatim}
## [1] TRUE
\end{verbatim}

Let's summarise the main points of this tutorial.

\begin{itemize}
\tightlist
\item
  The three most important logical operators are NOT, AND and OR.
\item
  In R, the NOT operator is the exclamation mark.
\item
  The AND operator is the ampersand.
\item
  The OR operator is the vertical bar.
\item
  Logical operators are often used to subset vectors or data frames.
\end{itemize}

\end{document}
