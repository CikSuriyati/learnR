% Options for packages loaded elsewhere
\PassOptionsToPackage{unicode}{hyperref}
\PassOptionsToPackage{hyphens}{url}
%
\documentclass[
]{article}
\usepackage{amsmath,amssymb}
\usepackage{iftex}
\ifPDFTeX
  \usepackage[T1]{fontenc}
  \usepackage[utf8]{inputenc}
  \usepackage{textcomp} % provide euro and other symbols
\else % if luatex or xetex
  \usepackage{unicode-math} % this also loads fontspec
  \defaultfontfeatures{Scale=MatchLowercase}
  \defaultfontfeatures[\rmfamily]{Ligatures=TeX,Scale=1}
\fi
\usepackage{lmodern}
\ifPDFTeX\else
  % xetex/luatex font selection
\fi
% Use upquote if available, for straight quotes in verbatim environments
\IfFileExists{upquote.sty}{\usepackage{upquote}}{}
\IfFileExists{microtype.sty}{% use microtype if available
  \usepackage[]{microtype}
  \UseMicrotypeSet[protrusion]{basicmath} % disable protrusion for tt fonts
}{}
\makeatletter
\@ifundefined{KOMAClassName}{% if non-KOMA class
  \IfFileExists{parskip.sty}{%
    \usepackage{parskip}
  }{% else
    \setlength{\parindent}{0pt}
    \setlength{\parskip}{6pt plus 2pt minus 1pt}}
}{% if KOMA class
  \KOMAoptions{parskip=half}}
\makeatother
\usepackage{xcolor}
\usepackage[margin=1in]{geometry}
\usepackage{color}
\usepackage{fancyvrb}
\newcommand{\VerbBar}{|}
\newcommand{\VERB}{\Verb[commandchars=\\\{\}]}
\DefineVerbatimEnvironment{Highlighting}{Verbatim}{commandchars=\\\{\}}
% Add ',fontsize=\small' for more characters per line
\usepackage{framed}
\definecolor{shadecolor}{RGB}{248,248,248}
\newenvironment{Shaded}{\begin{snugshade}}{\end{snugshade}}
\newcommand{\AlertTok}[1]{\textcolor[rgb]{0.94,0.16,0.16}{#1}}
\newcommand{\AnnotationTok}[1]{\textcolor[rgb]{0.56,0.35,0.01}{\textbf{\textit{#1}}}}
\newcommand{\AttributeTok}[1]{\textcolor[rgb]{0.13,0.29,0.53}{#1}}
\newcommand{\BaseNTok}[1]{\textcolor[rgb]{0.00,0.00,0.81}{#1}}
\newcommand{\BuiltInTok}[1]{#1}
\newcommand{\CharTok}[1]{\textcolor[rgb]{0.31,0.60,0.02}{#1}}
\newcommand{\CommentTok}[1]{\textcolor[rgb]{0.56,0.35,0.01}{\textit{#1}}}
\newcommand{\CommentVarTok}[1]{\textcolor[rgb]{0.56,0.35,0.01}{\textbf{\textit{#1}}}}
\newcommand{\ConstantTok}[1]{\textcolor[rgb]{0.56,0.35,0.01}{#1}}
\newcommand{\ControlFlowTok}[1]{\textcolor[rgb]{0.13,0.29,0.53}{\textbf{#1}}}
\newcommand{\DataTypeTok}[1]{\textcolor[rgb]{0.13,0.29,0.53}{#1}}
\newcommand{\DecValTok}[1]{\textcolor[rgb]{0.00,0.00,0.81}{#1}}
\newcommand{\DocumentationTok}[1]{\textcolor[rgb]{0.56,0.35,0.01}{\textbf{\textit{#1}}}}
\newcommand{\ErrorTok}[1]{\textcolor[rgb]{0.64,0.00,0.00}{\textbf{#1}}}
\newcommand{\ExtensionTok}[1]{#1}
\newcommand{\FloatTok}[1]{\textcolor[rgb]{0.00,0.00,0.81}{#1}}
\newcommand{\FunctionTok}[1]{\textcolor[rgb]{0.13,0.29,0.53}{\textbf{#1}}}
\newcommand{\ImportTok}[1]{#1}
\newcommand{\InformationTok}[1]{\textcolor[rgb]{0.56,0.35,0.01}{\textbf{\textit{#1}}}}
\newcommand{\KeywordTok}[1]{\textcolor[rgb]{0.13,0.29,0.53}{\textbf{#1}}}
\newcommand{\NormalTok}[1]{#1}
\newcommand{\OperatorTok}[1]{\textcolor[rgb]{0.81,0.36,0.00}{\textbf{#1}}}
\newcommand{\OtherTok}[1]{\textcolor[rgb]{0.56,0.35,0.01}{#1}}
\newcommand{\PreprocessorTok}[1]{\textcolor[rgb]{0.56,0.35,0.01}{\textit{#1}}}
\newcommand{\RegionMarkerTok}[1]{#1}
\newcommand{\SpecialCharTok}[1]{\textcolor[rgb]{0.81,0.36,0.00}{\textbf{#1}}}
\newcommand{\SpecialStringTok}[1]{\textcolor[rgb]{0.31,0.60,0.02}{#1}}
\newcommand{\StringTok}[1]{\textcolor[rgb]{0.31,0.60,0.02}{#1}}
\newcommand{\VariableTok}[1]{\textcolor[rgb]{0.00,0.00,0.00}{#1}}
\newcommand{\VerbatimStringTok}[1]{\textcolor[rgb]{0.31,0.60,0.02}{#1}}
\newcommand{\WarningTok}[1]{\textcolor[rgb]{0.56,0.35,0.01}{\textbf{\textit{#1}}}}
\usepackage{graphicx}
\makeatletter
\def\maxwidth{\ifdim\Gin@nat@width>\linewidth\linewidth\else\Gin@nat@width\fi}
\def\maxheight{\ifdim\Gin@nat@height>\textheight\textheight\else\Gin@nat@height\fi}
\makeatother
% Scale images if necessary, so that they will not overflow the page
% margins by default, and it is still possible to overwrite the defaults
% using explicit options in \includegraphics[width, height, ...]{}
\setkeys{Gin}{width=\maxwidth,height=\maxheight,keepaspectratio}
% Set default figure placement to htbp
\makeatletter
\def\fps@figure{htbp}
\makeatother
\setlength{\emergencystretch}{3em} % prevent overfull lines
\providecommand{\tightlist}{%
  \setlength{\itemsep}{0pt}\setlength{\parskip}{0pt}}
\setcounter{secnumdepth}{-\maxdimen} % remove section numbering
\ifLuaTeX
  \usepackage{selnolig}  % disable illegal ligatures
\fi
\IfFileExists{bookmark.sty}{\usepackage{bookmark}}{\usepackage{hyperref}}
\IfFileExists{xurl.sty}{\usepackage{xurl}}{} % add URL line breaks if available
\urlstyle{same}
\hypersetup{
  pdftitle={Topic 1: Introduction to R language},
  hidelinks,
  pdfcreator={LaTeX via pandoc}}

\title{Topic 1: Introduction to R language}
\author{}
\date{\vspace{-2.5em}2023-08-22}

\begin{document}
\maketitle

In this topic in week 2 , you will learn about :

\begin{itemize}
\tightlist
\item
  Data frames and list
\item
  Extracting elements from vectors
\end{itemize}

The readings in this tutorial follow
\href{http://r4ds.had.co.nz/}{\emph{R for Data Science}}, section 5.2.

\hypertarget{section}{%
\section{}\label{section}}

\hypertarget{objects-in-r}{%
\subsection{Objects in R}\label{objects-in-r}}

\textbf{Objectives}

\begin{itemize}
\tightlist
\item
  To introduce data types in R and show how these data types are used in
  data structures.
\item
  Learn how to create vectors of different types.
\item
  Be able to check the type of vector.
\item
  Learn about missing data and other special values.
\item
  Get familiar with the different data structures (lists, matrices, data
  frames).
\end{itemize}

\textcolor{blue}{**Everything in R is an object.**}

R has 5 basic data types.

\begin{itemize}
\tightlist
\item
  character
\item
  numeric (real or decimal)
\item
  integer
\item
  logical
\item
  complex
\end{itemize}

A variable can store different types of values such as numbers,
characters etc. The
\textcolor{blue}{**variables are assigned with R-Objects**} and the
\textcolor{blue}{**data type of the R-object**} becomes the
\textcolor{blue}{**data type of the variable.**}

Elements of these data types may be combined to form data structures,
such as atomic vectors. When we call a vector atomic, we mean that the
vector only holds data of a single data type.

R provides many functions to examine features of vectors and other
objects, for example

\textbf{class( )} - what kind of object is it (high-level)?

\begin{Shaded}
\begin{Highlighting}[]
\NormalTok{x }\OtherTok{\textless{}{-}}\NormalTok{ 3L}
\FunctionTok{class}\NormalTok{(x)}
\end{Highlighting}
\end{Shaded}

\begin{verbatim}
## [1] "integer"
\end{verbatim}

\textbf{typeof( )} - what is the object's data type (low-level)?

\begin{Shaded}
\begin{Highlighting}[]
\NormalTok{x }\OtherTok{\textless{}{-}}\NormalTok{ 3L}
\FunctionTok{typeof}\NormalTok{(x)}
\end{Highlighting}
\end{Shaded}

\begin{verbatim}
## [1] "integer"
\end{verbatim}

\textbf{length( )} - how long is it? What about two dimensional objects?

\begin{Shaded}
\begin{Highlighting}[]
\NormalTok{x }\OtherTok{\textless{}{-}}\NormalTok{ 3L}
\FunctionTok{length}\NormalTok{(x)}
\end{Highlighting}
\end{Shaded}

\begin{verbatim}
## [1] 1
\end{verbatim}

\textbf{attributes( )} - does it have any metadata?

\begin{Shaded}
\begin{Highlighting}[]
\NormalTok{x }\OtherTok{\textless{}{-}}\NormalTok{ 3L}
\FunctionTok{attributes}\NormalTok{(x)}
\end{Highlighting}
\end{Shaded}

\begin{verbatim}
## NULL
\end{verbatim}

Below are examples of atomic character vectors, numeric vectors, integer
vectors, etc.

\begin{itemize}
\item
  character: \textless{}``a''\textgreater, ``abc''
\item
  numeric: 3, 13.5
\item
  integer: (the L tells R to store this as an integer)
\item
  logical: TRUE, FALSE
\item
  complex: 1+3i (complex numbers with real and imaginary parts)
\end{itemize}

R has many data structures. The frequently used ones are:

\begin{itemize}
\tightlist
\item
  Vectors
\item
  Lists
\item
  Matrices
\item
  Arrays
\item
  Factors
\item
  Data Frames
\end{itemize}

\hypertarget{vectors}{%
\subsection{Vectors}\label{vectors}}

A vector is a basic data structure which plays an important role in R
programming. In R,
\textcolor{blue}{a sequence of elements which share the same data type is known as vector}.
A vector supports logical, integer, double, character, complex, or raw
data type.

\begin{Shaded}
\begin{Highlighting}[]
\CommentTok{\# Atomic vector of type character.}
\FunctionTok{print}\NormalTok{(}\StringTok{"ali"}\NormalTok{);}
\end{Highlighting}
\end{Shaded}

\begin{verbatim}
## [1] "ali"
\end{verbatim}

\begin{Shaded}
\begin{Highlighting}[]
\CommentTok{\# Atomic vector of type double.}
\FunctionTok{print}\NormalTok{(}\FloatTok{2.5}\NormalTok{)}
\end{Highlighting}
\end{Shaded}

\begin{verbatim}
## [1] 2.5
\end{verbatim}

\begin{Shaded}
\begin{Highlighting}[]
\CommentTok{\# Atomic vector of type integer.}
\FunctionTok{print}\NormalTok{(3L)}
\end{Highlighting}
\end{Shaded}

\begin{verbatim}
## [1] 3
\end{verbatim}

\begin{Shaded}
\begin{Highlighting}[]
\CommentTok{\# Atomic vector of type logical.}
\FunctionTok{print}\NormalTok{(}\ConstantTok{TRUE}\NormalTok{)}
\end{Highlighting}
\end{Shaded}

\begin{verbatim}
## [1] TRUE
\end{verbatim}

\begin{Shaded}
\begin{Highlighting}[]
\CommentTok{\# Atomic vector of type complex.}
\FunctionTok{print}\NormalTok{(}\DecValTok{1}\SpecialCharTok{+}\NormalTok{3i)}
\end{Highlighting}
\end{Shaded}

\begin{verbatim}
## [1] 1+3i
\end{verbatim}

\begin{Shaded}
\begin{Highlighting}[]
\CommentTok{\# Atomic vector of type raw.}
\FunctionTok{print}\NormalTok{(}\FunctionTok{charToRaw}\NormalTok{(}\StringTok{\textquotesingle{}hello\textquotesingle{}}\NormalTok{))}
\end{Highlighting}
\end{Shaded}

\begin{verbatim}
## [1] 68 65 6c 6c 6f
\end{verbatim}

\textbf{1. Multiple Elements Vector}.

\textcolor{blue}{Using colon operator (:) with numeric data}

\begin{Shaded}
\begin{Highlighting}[]
\CommentTok{\# Creating a sequence from 4 to 23.}
\NormalTok{v }\OtherTok{\textless{}{-}} \DecValTok{4}\SpecialCharTok{:}\DecValTok{12}
\FunctionTok{print}\NormalTok{(v)}
\end{Highlighting}
\end{Shaded}

\begin{verbatim}
## [1]  4  5  6  7  8  9 10 11 12
\end{verbatim}

\begin{Shaded}
\begin{Highlighting}[]
\CommentTok{\# Creating a sequence from 3.6 to 9.6.}
\NormalTok{v }\OtherTok{\textless{}{-}} \FloatTok{3.6}\SpecialCharTok{:}\FloatTok{9.6}
\FunctionTok{print}\NormalTok{(v)}
\end{Highlighting}
\end{Shaded}

\begin{verbatim}
## [1] 3.6 4.6 5.6 6.6 7.6 8.6 9.6
\end{verbatim}

\begin{Shaded}
\begin{Highlighting}[]
\CommentTok{\# If the final element specified does not belong to the sequence then it is discarded.}
\NormalTok{v }\OtherTok{\textless{}{-}} \FloatTok{4.8}\SpecialCharTok{:}\FloatTok{10.4}
\FunctionTok{print}\NormalTok{(v)}
\end{Highlighting}
\end{Shaded}

\begin{verbatim}
## [1] 4.8 5.8 6.8 7.8 8.8 9.8
\end{verbatim}

\textcolor{blue}{Using sequence (seq.) operator}

\begin{Shaded}
\begin{Highlighting}[]
\CommentTok{\# Create vector with elements from 5 to 9 incrementing by 0.4.}
\FunctionTok{print}\NormalTok{(}\FunctionTok{seq}\NormalTok{(}\DecValTok{5}\NormalTok{, }\DecValTok{9}\NormalTok{, }\AttributeTok{by =} \FloatTok{0.4}\NormalTok{))}
\end{Highlighting}
\end{Shaded}

\begin{verbatim}
##  [1] 5.0 5.4 5.8 6.2 6.6 7.0 7.4 7.8 8.2 8.6 9.0
\end{verbatim}

\textcolor{blue}{Using the c( ) function}.

The non-character values are coerced to character type if one of the
elements is a character.

\begin{Shaded}
\begin{Highlighting}[]
\CommentTok{\# The logical and numeric values are converted to characters.}
\NormalTok{s }\OtherTok{\textless{}{-}} \FunctionTok{c}\NormalTok{(}\StringTok{\textquotesingle{}apple\textquotesingle{}}\NormalTok{,}\StringTok{\textquotesingle{}red\textquotesingle{}}\NormalTok{,}\DecValTok{4}\NormalTok{,}\ConstantTok{TRUE}\NormalTok{)}
\FunctionTok{print}\NormalTok{(s)}
\end{Highlighting}
\end{Shaded}

\begin{verbatim}
## [1] "apple" "red"   "4"     "TRUE"
\end{verbatim}

\textbf{2. Accessing Vector Elements}

Elements of a Vector are accessed using indexing. The \textbf{{[} {]}
brackets} are used for indexing. Indexing starts with position 1. Giving
a negative value in the index drops that element from
result.\textbf{TRUE, FALSE} or \textbf{0} and \textbf{1} can also be
used for indexing.

\begin{Shaded}
\begin{Highlighting}[]
\CommentTok{\# Accessing vector elements using position.}
\NormalTok{t }\OtherTok{\textless{}{-}} \FunctionTok{c}\NormalTok{(}\StringTok{"Sun"}\NormalTok{,}\StringTok{"Mon"}\NormalTok{,}\StringTok{"Tue"}\NormalTok{,}\StringTok{"Wed"}\NormalTok{,}\StringTok{"Thurs"}\NormalTok{,}\StringTok{"Fri"}\NormalTok{,}\StringTok{"Sat"}\NormalTok{)}
\NormalTok{u }\OtherTok{\textless{}{-}}\NormalTok{ t[}\FunctionTok{c}\NormalTok{(}\DecValTok{1}\NormalTok{,}\DecValTok{2}\NormalTok{,}\DecValTok{5}\NormalTok{)]}
\FunctionTok{print}\NormalTok{(u)}
\end{Highlighting}
\end{Shaded}

\begin{verbatim}
## [1] "Sun"   "Mon"   "Thurs"
\end{verbatim}

\begin{Shaded}
\begin{Highlighting}[]
\CommentTok{\# Accessing vector elements using logical indexing.}
\NormalTok{v }\OtherTok{\textless{}{-}}\NormalTok{ t[}\FunctionTok{c}\NormalTok{(}\ConstantTok{TRUE}\NormalTok{,}\ConstantTok{FALSE}\NormalTok{,}\ConstantTok{FALSE}\NormalTok{,}\ConstantTok{FALSE}\NormalTok{,}\ConstantTok{FALSE}\NormalTok{,}\ConstantTok{TRUE}\NormalTok{,}\ConstantTok{FALSE}\NormalTok{)]}
\FunctionTok{print}\NormalTok{(v)}
\end{Highlighting}
\end{Shaded}

\begin{verbatim}
## [1] "Sun" "Fri"
\end{verbatim}

\begin{Shaded}
\begin{Highlighting}[]
\CommentTok{\# Accessing vector elements using negative indexing.}
\NormalTok{x }\OtherTok{\textless{}{-}}\NormalTok{ t[}\FunctionTok{c}\NormalTok{(}\SpecialCharTok{{-}}\DecValTok{3}\NormalTok{,}\SpecialCharTok{{-}}\DecValTok{6}\NormalTok{)]}
\FunctionTok{print}\NormalTok{(x)}
\end{Highlighting}
\end{Shaded}

\begin{verbatim}
## [1] "Sun"   "Mon"   "Wed"   "Thurs" "Sat"
\end{verbatim}

\begin{Shaded}
\begin{Highlighting}[]
\CommentTok{\# Accessing vector elements using 0/1 indexing.}
\NormalTok{y }\OtherTok{\textless{}{-}}\NormalTok{ t[}\FunctionTok{c}\NormalTok{(}\DecValTok{0}\NormalTok{,}\DecValTok{0}\NormalTok{,}\DecValTok{0}\NormalTok{,}\DecValTok{0}\NormalTok{,}\DecValTok{0}\NormalTok{,}\DecValTok{0}\NormalTok{,}\DecValTok{1}\NormalTok{)]}
\FunctionTok{print}\NormalTok{(y)}
\end{Highlighting}
\end{Shaded}

\begin{verbatim}
## [1] "Sun"
\end{verbatim}

\textbf{3. Vector Manipulation}.

\textbf{Vector arithmetic}.

Two vectors of same length can be added, subtracted, multiplied or
divided giving the result as a vector output.

\begin{Shaded}
\begin{Highlighting}[]
\CommentTok{\# Create two vectors.}
\NormalTok{v1 }\OtherTok{\textless{}{-}} \FunctionTok{c}\NormalTok{(}\DecValTok{2}\NormalTok{,}\DecValTok{8}\NormalTok{,}\DecValTok{4}\NormalTok{,}\DecValTok{4}\NormalTok{,}\DecValTok{0}\NormalTok{,}\DecValTok{10}\NormalTok{)}
\NormalTok{v2 }\OtherTok{\textless{}{-}} \FunctionTok{c}\NormalTok{(}\DecValTok{3}\NormalTok{,}\DecValTok{11}\NormalTok{,}\DecValTok{0}\NormalTok{,}\DecValTok{7}\NormalTok{,}\DecValTok{1}\NormalTok{,}\DecValTok{3}\NormalTok{)}

\CommentTok{\# Vector addition.}
\NormalTok{add }\OtherTok{\textless{}{-}}\NormalTok{ v1}\SpecialCharTok{+}\NormalTok{v2}
\FunctionTok{print}\NormalTok{(add)}
\end{Highlighting}
\end{Shaded}

\begin{verbatim}
## [1]  5 19  4 11  1 13
\end{verbatim}

\begin{Shaded}
\begin{Highlighting}[]
\CommentTok{\# Vector subtraction.}
\NormalTok{sub }\OtherTok{\textless{}{-}}\NormalTok{ v1}\SpecialCharTok{{-}}\NormalTok{v2}
\FunctionTok{print}\NormalTok{(sub)}
\end{Highlighting}
\end{Shaded}

\begin{verbatim}
## [1] -1 -3  4 -3 -1  7
\end{verbatim}

\begin{Shaded}
\begin{Highlighting}[]
\CommentTok{\# Vector multiplication.}
\NormalTok{multi }\OtherTok{\textless{}{-}}\NormalTok{ v1}\SpecialCharTok{*}\NormalTok{v2}
\FunctionTok{print}\NormalTok{(multi)}
\end{Highlighting}
\end{Shaded}

\begin{verbatim}
## [1]  6 88  0 28  0 30
\end{verbatim}

\begin{Shaded}
\begin{Highlighting}[]
\CommentTok{\# Vector division.}
\NormalTok{divi }\OtherTok{\textless{}{-}}\NormalTok{ v1}\SpecialCharTok{/}\NormalTok{v2}
\FunctionTok{print}\NormalTok{(divi)}
\end{Highlighting}
\end{Shaded}

\begin{verbatim}
## [1] 0.6666667 0.7272727       Inf 0.5714286 0.0000000 3.3333333
\end{verbatim}

\textbf{Vector Element Recycling}

If we apply arithmetic operations to two vectors of unequal length, then
the elements of the shorter vector are recycled to complete the
operations.

\begin{Shaded}
\begin{Highlighting}[]
\NormalTok{v1 }\OtherTok{\textless{}{-}} \FunctionTok{c}\NormalTok{(}\DecValTok{3}\NormalTok{,}\DecValTok{8}\NormalTok{,}\DecValTok{4}\NormalTok{,}\DecValTok{5}\NormalTok{,}\DecValTok{0}\NormalTok{,}\DecValTok{11}\NormalTok{)}
\NormalTok{v2 }\OtherTok{\textless{}{-}} \FunctionTok{c}\NormalTok{(}\DecValTok{4}\NormalTok{,}\DecValTok{11}\NormalTok{)}
\CommentTok{\# V2 becomes c(4,11,4,11,4,11)}

\NormalTok{add.result }\OtherTok{\textless{}{-}}\NormalTok{ v1}\SpecialCharTok{+}\NormalTok{v2}
\FunctionTok{print}\NormalTok{(add.result)}
\end{Highlighting}
\end{Shaded}

\begin{verbatim}
## [1]  7 19  8 16  4 22
\end{verbatim}

\begin{Shaded}
\begin{Highlighting}[]
\NormalTok{sub.result }\OtherTok{\textless{}{-}}\NormalTok{ v1}\SpecialCharTok{{-}}\NormalTok{v2}
\FunctionTok{print}\NormalTok{(sub.result)}
\end{Highlighting}
\end{Shaded}

\begin{verbatim}
## [1] -1 -3  0 -6 -4  0
\end{verbatim}

\textbf{Vector Element Sorting}

Elements in a vector can be sorted using the sort() function.

\begin{Shaded}
\begin{Highlighting}[]
\NormalTok{v }\OtherTok{\textless{}{-}} \FunctionTok{c}\NormalTok{(}\DecValTok{3}\NormalTok{,}\DecValTok{8}\NormalTok{,}\DecValTok{4}\NormalTok{,}\DecValTok{5}\NormalTok{,}\DecValTok{0}\NormalTok{,}\DecValTok{11}\NormalTok{, }\SpecialCharTok{{-}}\DecValTok{9}\NormalTok{, }\DecValTok{304}\NormalTok{)}

\CommentTok{\# Sort the elements of the vector.}
\NormalTok{sort1 }\OtherTok{\textless{}{-}} \FunctionTok{sort}\NormalTok{(v)}
\FunctionTok{print}\NormalTok{(sort1)}
\end{Highlighting}
\end{Shaded}

\begin{verbatim}
## [1]  -9   0   3   4   5   8  11 304
\end{verbatim}

\begin{Shaded}
\begin{Highlighting}[]
\CommentTok{\# Sort the elements in the reverse order.}
\NormalTok{revsort }\OtherTok{\textless{}{-}} \FunctionTok{sort}\NormalTok{(v, }\AttributeTok{decreasing =} \ConstantTok{TRUE}\NormalTok{)}
\FunctionTok{print}\NormalTok{(revsort)}
\end{Highlighting}
\end{Shaded}

\begin{verbatim}
## [1] 304  11   8   5   4   3   0  -9
\end{verbatim}

\begin{Shaded}
\begin{Highlighting}[]
\CommentTok{\# Sorting character vectors.}
\NormalTok{v }\OtherTok{\textless{}{-}} \FunctionTok{c}\NormalTok{(}\StringTok{"Red"}\NormalTok{,}\StringTok{"Blue"}\NormalTok{,}\StringTok{"yellow"}\NormalTok{,}\StringTok{"violet"}\NormalTok{)}
\NormalTok{sort2 }\OtherTok{\textless{}{-}} \FunctionTok{sort}\NormalTok{(v)}
\FunctionTok{print}\NormalTok{(sort2)}
\end{Highlighting}
\end{Shaded}

\begin{verbatim}
## [1] "Blue"   "Red"    "violet" "yellow"
\end{verbatim}

\begin{Shaded}
\begin{Highlighting}[]
\CommentTok{\# Sorting character vectors in reverse order.}
\NormalTok{revsort }\OtherTok{\textless{}{-}} \FunctionTok{sort}\NormalTok{(v, }\AttributeTok{decreasing =} \ConstantTok{TRUE}\NormalTok{)}
\FunctionTok{print}\NormalTok{(revsort)}
\end{Highlighting}
\end{Shaded}

\begin{verbatim}
## [1] "yellow" "violet" "Red"    "Blue"
\end{verbatim}

\hypertarget{lists}{%
\subsection{Lists}\label{lists}}

Lists are the R objects which contain elements of different types like −
numbers, strings, vectors and another list inside it. A list can also
contain a matrix or a function as its elements. List is created using
\textbf{list()} function.

\textbf{1. Creating a List}

Following is an example to create a list containing strings, numbers,
vectors and a logical values.

\begin{Shaded}
\begin{Highlighting}[]
\CommentTok{\# Create a list containing strings, numbers, vectors and a logical}
\CommentTok{\# values.}
\NormalTok{list\_data }\OtherTok{\textless{}{-}} \FunctionTok{list}\NormalTok{(}\StringTok{"Red"}\NormalTok{, }\StringTok{"Green"}\NormalTok{, }\FunctionTok{c}\NormalTok{(}\DecValTok{21}\NormalTok{,}\DecValTok{32}\NormalTok{,}\DecValTok{11}\NormalTok{), }\ConstantTok{TRUE}\NormalTok{, }\FloatTok{51.23}\NormalTok{, }\FloatTok{119.1}\NormalTok{)}
\FunctionTok{print}\NormalTok{(list\_data)}
\end{Highlighting}
\end{Shaded}

\begin{verbatim}
## [[1]]
## [1] "Red"
## 
## [[2]]
## [1] "Green"
## 
## [[3]]
## [1] 21 32 11
## 
## [[4]]
## [1] TRUE
## 
## [[5]]
## [1] 51.23
## 
## [[6]]
## [1] 119.1
\end{verbatim}

\textbf{2. Naming List Elements}

The list elements can be given names and they can be accessed using
these names.

\begin{Shaded}
\begin{Highlighting}[]
\CommentTok{\# Create a list containing a vector, a matrix and a list.}
\NormalTok{list\_data }\OtherTok{\textless{}{-}} \FunctionTok{list}\NormalTok{(}\FunctionTok{c}\NormalTok{(}\StringTok{"Jan"}\NormalTok{,}\StringTok{"Feb"}\NormalTok{,}\StringTok{"Mar"}\NormalTok{), }\FunctionTok{matrix}\NormalTok{(}\FunctionTok{c}\NormalTok{(}\DecValTok{3}\NormalTok{,}\DecValTok{9}\NormalTok{,}\DecValTok{5}\NormalTok{,}\DecValTok{1}\NormalTok{,}\SpecialCharTok{{-}}\DecValTok{2}\NormalTok{,}\DecValTok{8}\NormalTok{), }\AttributeTok{nrow =} \DecValTok{2}\NormalTok{),}
   \FunctionTok{list}\NormalTok{(}\StringTok{"green"}\NormalTok{,}\FloatTok{12.3}\NormalTok{))}

\CommentTok{\# Give names to the elements in the list.}
\FunctionTok{names}\NormalTok{(list\_data) }\OtherTok{\textless{}{-}} \FunctionTok{c}\NormalTok{(}\StringTok{"1st Quarter"}\NormalTok{, }\StringTok{"A\_Matrix"}\NormalTok{, }\StringTok{"A Inner list"}\NormalTok{)}

\CommentTok{\# Show the list.}
\FunctionTok{print}\NormalTok{(list\_data)}
\end{Highlighting}
\end{Shaded}

\begin{verbatim}
## $`1st Quarter`
## [1] "Jan" "Feb" "Mar"
## 
## $A_Matrix
##      [,1] [,2] [,3]
## [1,]    3    5   -2
## [2,]    9    1    8
## 
## $`A Inner list`
## $`A Inner list`[[1]]
## [1] "green"
## 
## $`A Inner list`[[2]]
## [1] 12.3
\end{verbatim}

\textbf{3. Accessing List Elements}

Elements of the list can be accessed by the index of the element in the
list. In case of named lists it can also be accessed using the names.

We continue to use the list in the above example −

\begin{Shaded}
\begin{Highlighting}[]
\CommentTok{\# Create a list containing a vector, a matrix and a list.}
\NormalTok{list\_data }\OtherTok{\textless{}{-}} \FunctionTok{list}\NormalTok{(}\FunctionTok{c}\NormalTok{(}\StringTok{"Jan"}\NormalTok{,}\StringTok{"Feb"}\NormalTok{,}\StringTok{"Mar"}\NormalTok{), }\FunctionTok{matrix}\NormalTok{(}\FunctionTok{c}\NormalTok{(}\DecValTok{3}\NormalTok{,}\DecValTok{9}\NormalTok{,}\DecValTok{5}\NormalTok{,}\DecValTok{1}\NormalTok{,}\SpecialCharTok{{-}}\DecValTok{2}\NormalTok{,}\DecValTok{8}\NormalTok{), }\AttributeTok{nrow =} \DecValTok{2}\NormalTok{),}
   \FunctionTok{list}\NormalTok{(}\StringTok{"green"}\NormalTok{,}\FloatTok{12.3}\NormalTok{))}

\CommentTok{\# Give names to the elements in the list.}
\FunctionTok{names}\NormalTok{(list\_data) }\OtherTok{\textless{}{-}} \FunctionTok{c}\NormalTok{(}\StringTok{"1st Quarter"}\NormalTok{, }\StringTok{"A\_Matrix"}\NormalTok{, }\StringTok{"A Inner list"}\NormalTok{)}

\CommentTok{\# Access the first element of the list.}
\FunctionTok{print}\NormalTok{(list\_data[}\DecValTok{1}\NormalTok{])}
\end{Highlighting}
\end{Shaded}

\begin{verbatim}
## $`1st Quarter`
## [1] "Jan" "Feb" "Mar"
\end{verbatim}

\begin{Shaded}
\begin{Highlighting}[]
\CommentTok{\# Access the thrid element. As it is also a list, all its elements will be printed.}
\FunctionTok{print}\NormalTok{(list\_data[}\DecValTok{3}\NormalTok{])}
\end{Highlighting}
\end{Shaded}

\begin{verbatim}
## $`A Inner list`
## $`A Inner list`[[1]]
## [1] "green"
## 
## $`A Inner list`[[2]]
## [1] 12.3
\end{verbatim}

\begin{Shaded}
\begin{Highlighting}[]
\CommentTok{\# Access the list element using the name of the element.}
\FunctionTok{print}\NormalTok{(list\_data}\SpecialCharTok{$}\NormalTok{A\_Matrix)}
\end{Highlighting}
\end{Shaded}

\begin{verbatim}
##      [,1] [,2] [,3]
## [1,]    3    5   -2
## [2,]    9    1    8
\end{verbatim}

\textbf{4. Manipulating List Elements}

We can add, delete and update list elements as shown below. We can add
and delete elements only at the end of a list. But we can update any
element.

\begin{Shaded}
\begin{Highlighting}[]
\CommentTok{\# Create a list containing a vector, a matrix and a list.}
\NormalTok{list\_data }\OtherTok{\textless{}{-}} \FunctionTok{list}\NormalTok{(}\FunctionTok{c}\NormalTok{(}\StringTok{"Jan"}\NormalTok{,}\StringTok{"Feb"}\NormalTok{,}\StringTok{"Mar"}\NormalTok{), }\FunctionTok{matrix}\NormalTok{(}\FunctionTok{c}\NormalTok{(}\DecValTok{3}\NormalTok{,}\DecValTok{9}\NormalTok{,}\DecValTok{5}\NormalTok{,}\DecValTok{1}\NormalTok{,}\SpecialCharTok{{-}}\DecValTok{2}\NormalTok{,}\DecValTok{8}\NormalTok{), }\AttributeTok{nrow =} \DecValTok{2}\NormalTok{),}
   \FunctionTok{list}\NormalTok{(}\StringTok{"green"}\NormalTok{,}\FloatTok{12.3}\NormalTok{))}

\CommentTok{\# Give names to the elements in the list.}
\FunctionTok{names}\NormalTok{(list\_data) }\OtherTok{\textless{}{-}} \FunctionTok{c}\NormalTok{(}\StringTok{"1st Quarter"}\NormalTok{, }\StringTok{"A\_Matrix"}\NormalTok{, }\StringTok{"A Inner list"}\NormalTok{)}

\CommentTok{\# Add element at the end of the list.}
\NormalTok{list\_data[}\DecValTok{4}\NormalTok{] }\OtherTok{\textless{}{-}} \StringTok{"New element"}
\FunctionTok{print}\NormalTok{(list\_data[}\DecValTok{4}\NormalTok{])}
\end{Highlighting}
\end{Shaded}

\begin{verbatim}
## [[1]]
## [1] "New element"
\end{verbatim}

\begin{Shaded}
\begin{Highlighting}[]
\CommentTok{\# Remove the last element.}
\NormalTok{list\_data[}\DecValTok{4}\NormalTok{] }\OtherTok{\textless{}{-}} \ConstantTok{NULL}

\CommentTok{\# Print the 4th Element.}
\FunctionTok{print}\NormalTok{(list\_data[}\DecValTok{4}\NormalTok{])}
\end{Highlighting}
\end{Shaded}

\begin{verbatim}
## $<NA>
## NULL
\end{verbatim}

\begin{Shaded}
\begin{Highlighting}[]
\CommentTok{\# Update the 3rd Element.}
\NormalTok{list\_data[}\DecValTok{3}\NormalTok{] }\OtherTok{\textless{}{-}} \StringTok{"updated element"}
\FunctionTok{print}\NormalTok{(list\_data[}\DecValTok{3}\NormalTok{])}
\end{Highlighting}
\end{Shaded}

\begin{verbatim}
## $`A Inner list`
## [1] "updated element"
\end{verbatim}

\textbf{5. Merging Lists}

You can merge many lists into one list by placing all the lists inside
one list() function.

\begin{Shaded}
\begin{Highlighting}[]
\CommentTok{\# Create two lists.}
\NormalTok{list1 }\OtherTok{\textless{}{-}} \FunctionTok{list}\NormalTok{(}\DecValTok{1}\NormalTok{,}\DecValTok{2}\NormalTok{,}\DecValTok{3}\NormalTok{)}
\NormalTok{list2 }\OtherTok{\textless{}{-}} \FunctionTok{list}\NormalTok{(}\StringTok{"Sun"}\NormalTok{,}\StringTok{"Mon"}\NormalTok{,}\StringTok{"Tue"}\NormalTok{)}

\CommentTok{\# Merge the two lists.}
\NormalTok{merged.list }\OtherTok{\textless{}{-}} \FunctionTok{c}\NormalTok{(list1,list2)}

\CommentTok{\# Print the merged list.}
\FunctionTok{print}\NormalTok{(merged.list)}
\end{Highlighting}
\end{Shaded}

\begin{verbatim}
## [[1]]
## [1] 1
## 
## [[2]]
## [1] 2
## 
## [[3]]
## [1] 3
## 
## [[4]]
## [1] "Sun"
## 
## [[5]]
## [1] "Mon"
## 
## [[6]]
## [1] "Tue"
\end{verbatim}

\textbf{6. Converting List to Vector}

A list can be converted to a vector so that the elements of the vector
can be used for further manipulation. All the arithmetic operations on
vectors can be applied after the list is converted into vectors. To do
this conversion, we use the unlist() function. It takes the list as
input and produces a vector.

\begin{Shaded}
\begin{Highlighting}[]
\CommentTok{\# Create lists.}
\NormalTok{list1 }\OtherTok{\textless{}{-}} \FunctionTok{list}\NormalTok{(}\DecValTok{1}\SpecialCharTok{:}\DecValTok{5}\NormalTok{)}
\FunctionTok{print}\NormalTok{(list1)}
\end{Highlighting}
\end{Shaded}

\begin{verbatim}
## [[1]]
## [1] 1 2 3 4 5
\end{verbatim}

\begin{Shaded}
\begin{Highlighting}[]
\NormalTok{list2 }\OtherTok{\textless{}{-}}\FunctionTok{list}\NormalTok{(}\DecValTok{10}\SpecialCharTok{:}\DecValTok{14}\NormalTok{)}
\FunctionTok{print}\NormalTok{(list2)}
\end{Highlighting}
\end{Shaded}

\begin{verbatim}
## [[1]]
## [1] 10 11 12 13 14
\end{verbatim}

\begin{Shaded}
\begin{Highlighting}[]
\CommentTok{\# Convert the lists to vectors.}
\NormalTok{v1 }\OtherTok{\textless{}{-}} \FunctionTok{unlist}\NormalTok{(list1)}
\NormalTok{v2 }\OtherTok{\textless{}{-}} \FunctionTok{unlist}\NormalTok{(list2)}

\FunctionTok{print}\NormalTok{(v1)}
\end{Highlighting}
\end{Shaded}

\begin{verbatim}
## [1] 1 2 3 4 5
\end{verbatim}

\begin{Shaded}
\begin{Highlighting}[]
\FunctionTok{print}\NormalTok{(v2)}
\end{Highlighting}
\end{Shaded}

\begin{verbatim}
## [1] 10 11 12 13 14
\end{verbatim}

\begin{Shaded}
\begin{Highlighting}[]
\CommentTok{\# Now add the vectors}
\NormalTok{result }\OtherTok{\textless{}{-}}\NormalTok{ v1}\SpecialCharTok{+}\NormalTok{v2}
\FunctionTok{print}\NormalTok{(result)}
\end{Highlighting}
\end{Shaded}

\begin{verbatim}
## [1] 11 13 15 17 19
\end{verbatim}

\hypertarget{matrices}{%
\subsection{Matrices}\label{matrices}}

Matrices are the R objects in which the elements are arranged in a
two-dimensional rectangular layout. They contain elements of the same
atomic types. Though we can create a matrix containing only characters
or only logical values, they are not of much use. We use matrices
containing numeric elements to be used in mathematical calculations.

A Matrix is created using the \textbf{matrix()} function.

\textbf{Syntax}

The basic syntax for creating a matrix in R is −

\textbf{matrix(data, nrow, ncol, byrow, dimnames)}

Following is the description of the parameters used −

\begin{itemize}
\tightlist
\item
  data is the input vector which becomes the data elements of the
  matrix.
\item
  nrow is the number of rows to be created.
\item
  ncol is the number of columns to be created.
\item
  byrow is a logical clue. If TRUE then the input vector elements are
  arranged by row.
\item
  dimname is the names assigned to the rows and columns.
\end{itemize}

\textbf{Example}

Create a matrix taking a vector of numbers as input.

\begin{Shaded}
\begin{Highlighting}[]
\CommentTok{\# Elements are arranged sequentially by row.}
\NormalTok{M }\OtherTok{\textless{}{-}} \FunctionTok{matrix}\NormalTok{(}\FunctionTok{c}\NormalTok{(}\DecValTok{3}\SpecialCharTok{:}\DecValTok{14}\NormalTok{), }\AttributeTok{nrow =} \DecValTok{4}\NormalTok{, }\AttributeTok{byrow =} \ConstantTok{TRUE}\NormalTok{)}
\FunctionTok{print}\NormalTok{(M)}
\end{Highlighting}
\end{Shaded}

\begin{verbatim}
##      [,1] [,2] [,3]
## [1,]    3    4    5
## [2,]    6    7    8
## [3,]    9   10   11
## [4,]   12   13   14
\end{verbatim}

\begin{Shaded}
\begin{Highlighting}[]
\CommentTok{\# Elements are arranged sequentially by column.}
\NormalTok{N }\OtherTok{\textless{}{-}} \FunctionTok{matrix}\NormalTok{(}\FunctionTok{c}\NormalTok{(}\DecValTok{3}\SpecialCharTok{:}\DecValTok{14}\NormalTok{), }\AttributeTok{nrow =} \DecValTok{4}\NormalTok{, }\AttributeTok{byrow =} \ConstantTok{FALSE}\NormalTok{)}
\FunctionTok{print}\NormalTok{(N)}
\end{Highlighting}
\end{Shaded}

\begin{verbatim}
##      [,1] [,2] [,3]
## [1,]    3    7   11
## [2,]    4    8   12
## [3,]    5    9   13
## [4,]    6   10   14
\end{verbatim}

\begin{Shaded}
\begin{Highlighting}[]
\CommentTok{\# Define the column and row names.}
\NormalTok{rownames }\OtherTok{=} \FunctionTok{c}\NormalTok{(}\StringTok{"row1"}\NormalTok{, }\StringTok{"row2"}\NormalTok{, }\StringTok{"row3"}\NormalTok{, }\StringTok{"row4"}\NormalTok{)}
\NormalTok{colnames }\OtherTok{=} \FunctionTok{c}\NormalTok{(}\StringTok{"col1"}\NormalTok{, }\StringTok{"col2"}\NormalTok{, }\StringTok{"col3"}\NormalTok{)}

\NormalTok{P }\OtherTok{\textless{}{-}} \FunctionTok{matrix}\NormalTok{(}\FunctionTok{c}\NormalTok{(}\DecValTok{3}\SpecialCharTok{:}\DecValTok{14}\NormalTok{), }\AttributeTok{nrow =} \DecValTok{4}\NormalTok{, }\AttributeTok{byrow =} \ConstantTok{TRUE}\NormalTok{, }\AttributeTok{dimnames =} \FunctionTok{list}\NormalTok{(rownames, colnames))}
\FunctionTok{print}\NormalTok{(P)}
\end{Highlighting}
\end{Shaded}

\begin{verbatim}
##      col1 col2 col3
## row1    3    4    5
## row2    6    7    8
## row3    9   10   11
## row4   12   13   14
\end{verbatim}

\textcolor{blue}{**Accessing Elements of a Matrix**}

Elements of a matrix can be accessed by using the column and row index
of the element. We consider the matrix P above to find the specific
elements below.

\begin{Shaded}
\begin{Highlighting}[]
\CommentTok{\# Define the column and row names.}
\NormalTok{rownames }\OtherTok{=} \FunctionTok{c}\NormalTok{(}\StringTok{"row1"}\NormalTok{, }\StringTok{"row2"}\NormalTok{, }\StringTok{"row3"}\NormalTok{, }\StringTok{"row4"}\NormalTok{)}
\NormalTok{colnames }\OtherTok{=} \FunctionTok{c}\NormalTok{(}\StringTok{"col1"}\NormalTok{, }\StringTok{"col2"}\NormalTok{, }\StringTok{"col3"}\NormalTok{)}

\CommentTok{\# Create the matrix.}
\NormalTok{P }\OtherTok{\textless{}{-}} \FunctionTok{matrix}\NormalTok{(}\FunctionTok{c}\NormalTok{(}\DecValTok{3}\SpecialCharTok{:}\DecValTok{14}\NormalTok{), }\AttributeTok{nrow =} \DecValTok{4}\NormalTok{, }\AttributeTok{byrow =} \ConstantTok{TRUE}\NormalTok{, }\AttributeTok{dimnames =} \FunctionTok{list}\NormalTok{(rownames, colnames))}

\CommentTok{\# Access the element at 3rd column and 1st row.}
\FunctionTok{print}\NormalTok{(P[}\DecValTok{1}\NormalTok{,}\DecValTok{3}\NormalTok{])}
\end{Highlighting}
\end{Shaded}

\begin{verbatim}
## [1] 5
\end{verbatim}

\begin{Shaded}
\begin{Highlighting}[]
\CommentTok{\# Access the element at 2nd column and 4th row.}
\FunctionTok{print}\NormalTok{(P[}\DecValTok{4}\NormalTok{,}\DecValTok{2}\NormalTok{])}
\end{Highlighting}
\end{Shaded}

\begin{verbatim}
## [1] 13
\end{verbatim}

\begin{Shaded}
\begin{Highlighting}[]
\CommentTok{\# Access only the  2nd row.}
\FunctionTok{print}\NormalTok{(P[}\DecValTok{2}\NormalTok{,])}
\end{Highlighting}
\end{Shaded}

\begin{verbatim}
## col1 col2 col3 
##    6    7    8
\end{verbatim}

\begin{Shaded}
\begin{Highlighting}[]
\CommentTok{\# Access only the 3rd column.}
\FunctionTok{print}\NormalTok{(P[,}\DecValTok{3}\NormalTok{])}
\end{Highlighting}
\end{Shaded}

\begin{verbatim}
## row1 row2 row3 row4 
##    5    8   11   14
\end{verbatim}

\textcolor{blue}{**Matrix Computations**}

Various mathematical operations are performed on the matrices using the
R operators. The result of the operation is also a matrix.

The dimensions (number of rows and columns) should be same for the
matrices involved in the operation.

\textcolor{blue}{**1. Matrix Addition & Subtraction**}

\begin{Shaded}
\begin{Highlighting}[]
\CommentTok{\# Create two 2x3 matrices.}
\NormalTok{matrix1 }\OtherTok{\textless{}{-}} \FunctionTok{matrix}\NormalTok{(}\FunctionTok{c}\NormalTok{(}\DecValTok{3}\NormalTok{, }\DecValTok{9}\NormalTok{, }\SpecialCharTok{{-}}\DecValTok{1}\NormalTok{, }\DecValTok{4}\NormalTok{, }\DecValTok{2}\NormalTok{, }\DecValTok{6}\NormalTok{), }\AttributeTok{nrow =} \DecValTok{2}\NormalTok{)}
\FunctionTok{print}\NormalTok{(matrix1)}
\end{Highlighting}
\end{Shaded}

\begin{verbatim}
##      [,1] [,2] [,3]
## [1,]    3   -1    2
## [2,]    9    4    6
\end{verbatim}

\begin{Shaded}
\begin{Highlighting}[]
\NormalTok{matrix2 }\OtherTok{\textless{}{-}} \FunctionTok{matrix}\NormalTok{(}\FunctionTok{c}\NormalTok{(}\DecValTok{5}\NormalTok{, }\DecValTok{2}\NormalTok{, }\DecValTok{0}\NormalTok{, }\DecValTok{9}\NormalTok{, }\DecValTok{3}\NormalTok{, }\DecValTok{4}\NormalTok{), }\AttributeTok{nrow =} \DecValTok{2}\NormalTok{)}
\FunctionTok{print}\NormalTok{(matrix2)}
\end{Highlighting}
\end{Shaded}

\begin{verbatim}
##      [,1] [,2] [,3]
## [1,]    5    0    3
## [2,]    2    9    4
\end{verbatim}

\begin{Shaded}
\begin{Highlighting}[]
\CommentTok{\# Add the matrices.}
\NormalTok{result }\OtherTok{\textless{}{-}}\NormalTok{ matrix1 }\SpecialCharTok{+}\NormalTok{ matrix2}
\FunctionTok{cat}\NormalTok{(}\StringTok{"Result of addition"}\NormalTok{,}\StringTok{"}\SpecialCharTok{\textbackslash{}n}\StringTok{"}\NormalTok{)}
\end{Highlighting}
\end{Shaded}

\begin{verbatim}
## Result of addition
\end{verbatim}

\begin{Shaded}
\begin{Highlighting}[]
\FunctionTok{print}\NormalTok{(result)}
\end{Highlighting}
\end{Shaded}

\begin{verbatim}
##      [,1] [,2] [,3]
## [1,]    8   -1    5
## [2,]   11   13   10
\end{verbatim}

\begin{Shaded}
\begin{Highlighting}[]
\CommentTok{\# Subtract the matrices}
\NormalTok{result }\OtherTok{\textless{}{-}}\NormalTok{ matrix1 }\SpecialCharTok{{-}}\NormalTok{ matrix2}
\FunctionTok{cat}\NormalTok{(}\StringTok{"Result of subtraction"}\NormalTok{,}\StringTok{"}\SpecialCharTok{\textbackslash{}n}\StringTok{"}\NormalTok{)}
\end{Highlighting}
\end{Shaded}

\begin{verbatim}
## Result of subtraction
\end{verbatim}

\begin{Shaded}
\begin{Highlighting}[]
\FunctionTok{print}\NormalTok{(result)}
\end{Highlighting}
\end{Shaded}

\begin{verbatim}
##      [,1] [,2] [,3]
## [1,]   -2   -1   -1
## [2,]    7   -5    2
\end{verbatim}

\textcolor{blue}{**2. Matrix Multiplication & Division**}

\begin{Shaded}
\begin{Highlighting}[]
\CommentTok{\# Create two 2x3 matrices.}
\NormalTok{matrix1 }\OtherTok{\textless{}{-}} \FunctionTok{matrix}\NormalTok{(}\FunctionTok{c}\NormalTok{(}\DecValTok{3}\NormalTok{, }\DecValTok{9}\NormalTok{, }\SpecialCharTok{{-}}\DecValTok{1}\NormalTok{, }\DecValTok{4}\NormalTok{, }\DecValTok{2}\NormalTok{, }\DecValTok{6}\NormalTok{), }\AttributeTok{nrow =} \DecValTok{2}\NormalTok{)}
\FunctionTok{print}\NormalTok{(matrix1)}
\end{Highlighting}
\end{Shaded}

\begin{verbatim}
##      [,1] [,2] [,3]
## [1,]    3   -1    2
## [2,]    9    4    6
\end{verbatim}

\begin{Shaded}
\begin{Highlighting}[]
\NormalTok{matrix2 }\OtherTok{\textless{}{-}} \FunctionTok{matrix}\NormalTok{(}\FunctionTok{c}\NormalTok{(}\DecValTok{5}\NormalTok{, }\DecValTok{2}\NormalTok{, }\DecValTok{0}\NormalTok{, }\DecValTok{9}\NormalTok{, }\DecValTok{3}\NormalTok{, }\DecValTok{4}\NormalTok{), }\AttributeTok{nrow =} \DecValTok{2}\NormalTok{)}
\FunctionTok{print}\NormalTok{(matrix2)}
\end{Highlighting}
\end{Shaded}

\begin{verbatim}
##      [,1] [,2] [,3]
## [1,]    5    0    3
## [2,]    2    9    4
\end{verbatim}

\begin{Shaded}
\begin{Highlighting}[]
\CommentTok{\# Multiply the matrices.}
\NormalTok{result }\OtherTok{\textless{}{-}}\NormalTok{ matrix1 }\SpecialCharTok{*}\NormalTok{ matrix2}
\FunctionTok{cat}\NormalTok{(}\StringTok{"Result of multiplication"}\NormalTok{,}\StringTok{"}\SpecialCharTok{\textbackslash{}n}\StringTok{"}\NormalTok{)}
\end{Highlighting}
\end{Shaded}

\begin{verbatim}
## Result of multiplication
\end{verbatim}

\begin{Shaded}
\begin{Highlighting}[]
\FunctionTok{print}\NormalTok{(result)}
\end{Highlighting}
\end{Shaded}

\begin{verbatim}
##      [,1] [,2] [,3]
## [1,]   15    0    6
## [2,]   18   36   24
\end{verbatim}

\begin{Shaded}
\begin{Highlighting}[]
\CommentTok{\# Divide the matrices}
\NormalTok{result }\OtherTok{\textless{}{-}}\NormalTok{ matrix1 }\SpecialCharTok{/}\NormalTok{ matrix2}
\FunctionTok{cat}\NormalTok{(}\StringTok{"Result of division"}\NormalTok{,}\StringTok{"}\SpecialCharTok{\textbackslash{}n}\StringTok{"}\NormalTok{)}
\end{Highlighting}
\end{Shaded}

\begin{verbatim}
## Result of division
\end{verbatim}

\begin{Shaded}
\begin{Highlighting}[]
\FunctionTok{print}\NormalTok{(result)}
\end{Highlighting}
\end{Shaded}

\begin{verbatim}
##      [,1]      [,2]      [,3]
## [1,]  0.6      -Inf 0.6666667
## [2,]  4.5 0.4444444 1.5000000
\end{verbatim}

\hypertarget{arrays}{%
\subsection{Arrays}\label{arrays}}

Arrays are the R data objects which can store data in more than two
dimensions. For example − If we create an array of dimension (2, 3, 4)
then it creates 4 rectangular matrices each with 2 rows and 3 columns.
Arrays can store only data type.

An array is created using the array() function. It takes vectors as
input and uses the values in the dim parameter to create an array.

\textbf{Example}

The following example creates an array of two 3x3 matrices each with 3
rows and 3 columns.

\begin{Shaded}
\begin{Highlighting}[]
\CommentTok{\# Create two vectors of different lengths.}
\NormalTok{vector1 }\OtherTok{\textless{}{-}} \FunctionTok{c}\NormalTok{(}\DecValTok{5}\NormalTok{,}\DecValTok{9}\NormalTok{,}\DecValTok{3}\NormalTok{)}
\NormalTok{vector2 }\OtherTok{\textless{}{-}} \FunctionTok{c}\NormalTok{(}\DecValTok{10}\NormalTok{,}\DecValTok{11}\NormalTok{,}\DecValTok{12}\NormalTok{,}\DecValTok{13}\NormalTok{,}\DecValTok{14}\NormalTok{,}\DecValTok{15}\NormalTok{)}

\CommentTok{\# Take these vectors as input to the array.}
\NormalTok{result }\OtherTok{\textless{}{-}} \FunctionTok{array}\NormalTok{(}\FunctionTok{c}\NormalTok{(vector1,vector2),}\AttributeTok{dim =} \FunctionTok{c}\NormalTok{(}\DecValTok{3}\NormalTok{,}\DecValTok{3}\NormalTok{,}\DecValTok{2}\NormalTok{))}
\FunctionTok{print}\NormalTok{(result)}
\end{Highlighting}
\end{Shaded}

\begin{verbatim}
## , , 1
## 
##      [,1] [,2] [,3]
## [1,]    5   10   13
## [2,]    9   11   14
## [3,]    3   12   15
## 
## , , 2
## 
##      [,1] [,2] [,3]
## [1,]    5   10   13
## [2,]    9   11   14
## [3,]    3   12   15
\end{verbatim}

\textcolor{blue}{**1. Naming Columns and Rows**}

We can give names to the rows, columns and matrices in the array by
using the dimnames parameter.

\begin{Shaded}
\begin{Highlighting}[]
\CommentTok{\# Create two vectors of different lengths.}
\NormalTok{vector1 }\OtherTok{\textless{}{-}} \FunctionTok{c}\NormalTok{(}\DecValTok{5}\NormalTok{,}\DecValTok{9}\NormalTok{,}\DecValTok{3}\NormalTok{)}
\NormalTok{vector2 }\OtherTok{\textless{}{-}} \FunctionTok{c}\NormalTok{(}\DecValTok{10}\NormalTok{,}\DecValTok{11}\NormalTok{,}\DecValTok{12}\NormalTok{,}\DecValTok{13}\NormalTok{,}\DecValTok{14}\NormalTok{,}\DecValTok{15}\NormalTok{)}
\NormalTok{column.names }\OtherTok{\textless{}{-}} \FunctionTok{c}\NormalTok{(}\StringTok{"COL1"}\NormalTok{,}\StringTok{"COL2"}\NormalTok{,}\StringTok{"COL3"}\NormalTok{)}
\NormalTok{row.names }\OtherTok{\textless{}{-}} \FunctionTok{c}\NormalTok{(}\StringTok{"ROW1"}\NormalTok{,}\StringTok{"ROW2"}\NormalTok{,}\StringTok{"ROW3"}\NormalTok{)}
\NormalTok{matrix.names }\OtherTok{\textless{}{-}} \FunctionTok{c}\NormalTok{(}\StringTok{"Matrix1"}\NormalTok{,}\StringTok{"Matrix2"}\NormalTok{)}

\CommentTok{\# Take these vectors as input to the array.}
\NormalTok{result }\OtherTok{\textless{}{-}} \FunctionTok{array}\NormalTok{(}\FunctionTok{c}\NormalTok{(vector1,vector2),}\AttributeTok{dim =} \FunctionTok{c}\NormalTok{(}\DecValTok{3}\NormalTok{,}\DecValTok{3}\NormalTok{,}\DecValTok{2}\NormalTok{),}\AttributeTok{dimnames =} \FunctionTok{list}\NormalTok{(row.names,column.names,}
\NormalTok{   matrix.names))}
\FunctionTok{print}\NormalTok{(result)}
\end{Highlighting}
\end{Shaded}

\begin{verbatim}
## , , Matrix1
## 
##      COL1 COL2 COL3
## ROW1    5   10   13
## ROW2    9   11   14
## ROW3    3   12   15
## 
## , , Matrix2
## 
##      COL1 COL2 COL3
## ROW1    5   10   13
## ROW2    9   11   14
## ROW3    3   12   15
\end{verbatim}

\textcolor{blue}{**2. Accessing Array Elements**}

\begin{Shaded}
\begin{Highlighting}[]
\CommentTok{\# Create two vectors of different lengths.}
\NormalTok{vector1 }\OtherTok{\textless{}{-}} \FunctionTok{c}\NormalTok{(}\DecValTok{5}\NormalTok{,}\DecValTok{9}\NormalTok{,}\DecValTok{3}\NormalTok{)}
\NormalTok{vector2 }\OtherTok{\textless{}{-}} \FunctionTok{c}\NormalTok{(}\DecValTok{10}\NormalTok{,}\DecValTok{11}\NormalTok{,}\DecValTok{12}\NormalTok{,}\DecValTok{13}\NormalTok{,}\DecValTok{14}\NormalTok{,}\DecValTok{15}\NormalTok{)}
\NormalTok{column.names }\OtherTok{\textless{}{-}} \FunctionTok{c}\NormalTok{(}\StringTok{"COL1"}\NormalTok{,}\StringTok{"COL2"}\NormalTok{,}\StringTok{"COL3"}\NormalTok{)}
\NormalTok{row.names }\OtherTok{\textless{}{-}} \FunctionTok{c}\NormalTok{(}\StringTok{"ROW1"}\NormalTok{,}\StringTok{"ROW2"}\NormalTok{,}\StringTok{"ROW3"}\NormalTok{)}
\NormalTok{matrix.names }\OtherTok{\textless{}{-}} \FunctionTok{c}\NormalTok{(}\StringTok{"Matrix1"}\NormalTok{,}\StringTok{"Matrix2"}\NormalTok{)}

\CommentTok{\# Take these vectors as input to the array.}
\NormalTok{result }\OtherTok{\textless{}{-}} \FunctionTok{array}\NormalTok{(}\FunctionTok{c}\NormalTok{(vector1,vector2),}\AttributeTok{dim =} \FunctionTok{c}\NormalTok{(}\DecValTok{3}\NormalTok{,}\DecValTok{3}\NormalTok{,}\DecValTok{2}\NormalTok{),}\AttributeTok{dimnames =} \FunctionTok{list}\NormalTok{(row.names,}
\NormalTok{   column.names, matrix.names))}

\CommentTok{\# Print the third row of the second matrix of the array.}
\FunctionTok{print}\NormalTok{(result[}\DecValTok{3}\NormalTok{,,}\DecValTok{2}\NormalTok{])}
\end{Highlighting}
\end{Shaded}

\begin{verbatim}
## COL1 COL2 COL3 
##    3   12   15
\end{verbatim}

\begin{Shaded}
\begin{Highlighting}[]
\CommentTok{\# Print the element in the 1st row and 3rd column of the 1st matrix.}
\FunctionTok{print}\NormalTok{(result[}\DecValTok{1}\NormalTok{,}\DecValTok{3}\NormalTok{,}\DecValTok{1}\NormalTok{])}
\end{Highlighting}
\end{Shaded}

\begin{verbatim}
## [1] 13
\end{verbatim}

\begin{Shaded}
\begin{Highlighting}[]
\CommentTok{\# Print the 2nd Matrix.}
\FunctionTok{print}\NormalTok{(result[,,}\DecValTok{2}\NormalTok{])}
\end{Highlighting}
\end{Shaded}

\begin{verbatim}
##      COL1 COL2 COL3
## ROW1    5   10   13
## ROW2    9   11   14
## ROW3    3   12   15
\end{verbatim}

\textcolor{blue}{**3. Manipulating Array Elements**}

As array is made up matrices in multiple dimensions, the operations on
elements of array are carried out by accessing elements of the matrices.

\begin{Shaded}
\begin{Highlighting}[]
\CommentTok{\# Create two vectors of different lengths.}
\NormalTok{vector1 }\OtherTok{\textless{}{-}} \FunctionTok{c}\NormalTok{(}\DecValTok{5}\NormalTok{,}\DecValTok{9}\NormalTok{,}\DecValTok{3}\NormalTok{)}
\NormalTok{vector2 }\OtherTok{\textless{}{-}} \FunctionTok{c}\NormalTok{(}\DecValTok{10}\NormalTok{,}\DecValTok{11}\NormalTok{,}\DecValTok{12}\NormalTok{,}\DecValTok{13}\NormalTok{,}\DecValTok{14}\NormalTok{,}\DecValTok{15}\NormalTok{)}

\CommentTok{\# Take these vectors as input to the array.}
\NormalTok{array1 }\OtherTok{\textless{}{-}} \FunctionTok{array}\NormalTok{(}\FunctionTok{c}\NormalTok{(vector1,vector2),}\AttributeTok{dim =} \FunctionTok{c}\NormalTok{(}\DecValTok{3}\NormalTok{,}\DecValTok{3}\NormalTok{,}\DecValTok{2}\NormalTok{))}

\CommentTok{\# Create two vectors of different lengths.}
\NormalTok{vector3 }\OtherTok{\textless{}{-}} \FunctionTok{c}\NormalTok{(}\DecValTok{9}\NormalTok{,}\DecValTok{1}\NormalTok{,}\DecValTok{0}\NormalTok{)}
\NormalTok{vector4 }\OtherTok{\textless{}{-}} \FunctionTok{c}\NormalTok{(}\DecValTok{6}\NormalTok{,}\DecValTok{0}\NormalTok{,}\DecValTok{11}\NormalTok{,}\DecValTok{3}\NormalTok{,}\DecValTok{14}\NormalTok{,}\DecValTok{1}\NormalTok{,}\DecValTok{2}\NormalTok{,}\DecValTok{6}\NormalTok{,}\DecValTok{9}\NormalTok{)}
\NormalTok{array2 }\OtherTok{\textless{}{-}} \FunctionTok{array}\NormalTok{(}\FunctionTok{c}\NormalTok{(vector1,vector2),}\AttributeTok{dim =} \FunctionTok{c}\NormalTok{(}\DecValTok{3}\NormalTok{,}\DecValTok{3}\NormalTok{,}\DecValTok{2}\NormalTok{))}

\CommentTok{\# create matrices from these arrays.}
\NormalTok{matrix1 }\OtherTok{\textless{}{-}}\NormalTok{ array1[,,}\DecValTok{2}\NormalTok{]}
\NormalTok{matrix2 }\OtherTok{\textless{}{-}}\NormalTok{ array2[,,}\DecValTok{2}\NormalTok{]}

\CommentTok{\# Add the matrices.}
\NormalTok{result }\OtherTok{\textless{}{-}}\NormalTok{ matrix1}\SpecialCharTok{+}\NormalTok{matrix2}
\FunctionTok{print}\NormalTok{(result)}
\end{Highlighting}
\end{Shaded}

\begin{verbatim}
##      [,1] [,2] [,3]
## [1,]   10   20   26
## [2,]   18   22   28
## [3,]    6   24   30
\end{verbatim}

\textcolor{blue}{**4. Calculations Across Array Elements**}

We can do calculations across the elements in an array using the apply()
function.

\textbf{Syntax}

\textbf{apply(x, margin, fun)}

Following is the description of the parameters used −

\begin{itemize}
\tightlist
\item
  x is an array.
\item
  margin is the name of the data set used.
\item
  fun is the function to be applied across the elements of the array.
\end{itemize}

\textbf{Example}

We use the \textbf{apply()} function below to calculate the sum of the
elements in the rows of an array across all the matrices

\begin{Shaded}
\begin{Highlighting}[]
\CommentTok{\# Create two vectors of different lengths.}
\NormalTok{vector1 }\OtherTok{\textless{}{-}} \FunctionTok{c}\NormalTok{(}\DecValTok{5}\NormalTok{,}\DecValTok{9}\NormalTok{,}\DecValTok{3}\NormalTok{)}
\NormalTok{vector2 }\OtherTok{\textless{}{-}} \FunctionTok{c}\NormalTok{(}\DecValTok{10}\NormalTok{,}\DecValTok{11}\NormalTok{,}\DecValTok{12}\NormalTok{,}\DecValTok{13}\NormalTok{,}\DecValTok{14}\NormalTok{,}\DecValTok{15}\NormalTok{)}

\CommentTok{\# Take these vectors as input to the array.}
\NormalTok{new.array }\OtherTok{\textless{}{-}} \FunctionTok{array}\NormalTok{(}\FunctionTok{c}\NormalTok{(vector1,vector2),}\AttributeTok{dim =} \FunctionTok{c}\NormalTok{(}\DecValTok{3}\NormalTok{,}\DecValTok{3}\NormalTok{,}\DecValTok{2}\NormalTok{))}
\FunctionTok{print}\NormalTok{(new.array)}
\end{Highlighting}
\end{Shaded}

\begin{verbatim}
## , , 1
## 
##      [,1] [,2] [,3]
## [1,]    5   10   13
## [2,]    9   11   14
## [3,]    3   12   15
## 
## , , 2
## 
##      [,1] [,2] [,3]
## [1,]    5   10   13
## [2,]    9   11   14
## [3,]    3   12   15
\end{verbatim}

\begin{Shaded}
\begin{Highlighting}[]
\CommentTok{\# Use apply to calculate the sum of the rows across all the matrices.}
\NormalTok{result }\OtherTok{\textless{}{-}} \FunctionTok{apply}\NormalTok{(new.array, }\FunctionTok{c}\NormalTok{(}\DecValTok{1}\NormalTok{), sum)}
\FunctionTok{print}\NormalTok{(result)}
\end{Highlighting}
\end{Shaded}

\begin{verbatim}
## [1] 56 68 60
\end{verbatim}

\hypertarget{data-frame}{%
\subsection{Data Frame}\label{data-frame}}

A data frame is a table or a two-dimensional array-like structure in
which each column contains values of one variable and each row contains
one set of values from each column.

Following are the characteristics of a data frame.

\begin{itemize}
\tightlist
\item
  The column names should be non-empty.
\item
  The row names should be unique.
\item
  The data stored in a data frame can be of numeric, factor or character
  type.
\item
  Each column should contain same number of data items.
\end{itemize}

\textcolor{blue}{**1. Create Data Frame**}

\begin{Shaded}
\begin{Highlighting}[]
\CommentTok{\# Create the data frame.}
\NormalTok{student.data }\OtherTok{\textless{}{-}} \FunctionTok{data.frame}\NormalTok{(}
   \AttributeTok{std\_id =} \FunctionTok{c}\NormalTok{ (}\DecValTok{1}\SpecialCharTok{:}\DecValTok{5}\NormalTok{), }
   \AttributeTok{std\_name =} \FunctionTok{c}\NormalTok{(}\StringTok{"Ali"}\NormalTok{,}\StringTok{"Abu"}\NormalTok{,}\StringTok{"Ahmad"}\NormalTok{,}\StringTok{"Siti"}\NormalTok{,}\StringTok{"Emma"}\NormalTok{),}
   \AttributeTok{pocketmoney =} \FunctionTok{c}\NormalTok{(}\FloatTok{623.3}\NormalTok{,}\FloatTok{515.2}\NormalTok{,}\FloatTok{611.0}\NormalTok{,}\FloatTok{729.0}\NormalTok{,}\FloatTok{843.25}\NormalTok{), }
   
   \AttributeTok{start\_date =} \FunctionTok{as.Date}\NormalTok{(}\FunctionTok{c}\NormalTok{(}\StringTok{"2012{-}01{-}01"}\NormalTok{, }\StringTok{"2013{-}09{-}23"}\NormalTok{, }\StringTok{"2014{-}11{-}15"}\NormalTok{, }\StringTok{"2014{-}05{-}11"}\NormalTok{,}
      \StringTok{"2015{-}03{-}27"}\NormalTok{)),}
  
   \AttributeTok{stringsAsFactors =} \ConstantTok{FALSE}
\NormalTok{)}
\CommentTok{\# Print the data frame.         }
\FunctionTok{print}\NormalTok{(student.data) }
\end{Highlighting}
\end{Shaded}

\begin{verbatim}
##   std_id std_name pocketmoney start_date
## 1      1      Ali      623.30 2012-01-01
## 2      2      Abu      515.20 2013-09-23
## 3      3    Ahmad      611.00 2014-11-15
## 4      4     Siti      729.00 2014-05-11
## 5      5     Emma      843.25 2015-03-27
\end{verbatim}

\textcolor{blue}{**2. Get the Structure of the Data Frame**}

The structure of the data frame can be seen by using str() function.

\begin{Shaded}
\begin{Highlighting}[]
\CommentTok{\# Create the data frame.}
\NormalTok{student.data }\OtherTok{\textless{}{-}} \FunctionTok{data.frame}\NormalTok{(}
   \AttributeTok{std\_id =} \FunctionTok{c}\NormalTok{ (}\DecValTok{1}\SpecialCharTok{:}\DecValTok{5}\NormalTok{), }
   \AttributeTok{std\_name =} \FunctionTok{c}\NormalTok{(}\StringTok{"Ali"}\NormalTok{,}\StringTok{"Abu"}\NormalTok{,}\StringTok{"Ahmad"}\NormalTok{,}\StringTok{"Siti"}\NormalTok{,}\StringTok{"Emma"}\NormalTok{),}
   \AttributeTok{pocketmoney =} \FunctionTok{c}\NormalTok{(}\FloatTok{623.3}\NormalTok{,}\FloatTok{515.2}\NormalTok{,}\FloatTok{611.0}\NormalTok{,}\FloatTok{729.0}\NormalTok{,}\FloatTok{843.25}\NormalTok{), }
   
   \AttributeTok{start\_date =} \FunctionTok{as.Date}\NormalTok{(}\FunctionTok{c}\NormalTok{(}\StringTok{"2012{-}01{-}01"}\NormalTok{, }\StringTok{"2013{-}09{-}23"}\NormalTok{, }\StringTok{"2014{-}11{-}15"}\NormalTok{, }\StringTok{"2014{-}05{-}11"}\NormalTok{,}
      \StringTok{"2015{-}03{-}27"}\NormalTok{)),}

   \AttributeTok{stringsAsFactors =} \ConstantTok{FALSE}
\NormalTok{)}
\CommentTok{\# Get the structure of the data frame.}
\FunctionTok{str}\NormalTok{(student.data)}
\end{Highlighting}
\end{Shaded}

\begin{verbatim}
## 'data.frame':    5 obs. of  4 variables:
##  $ std_id     : int  1 2 3 4 5
##  $ std_name   : chr  "Ali" "Abu" "Ahmad" "Siti" ...
##  $ pocketmoney: num  623 515 611 729 843
##  $ start_date : Date, format: "2012-01-01" "2013-09-23" ...
\end{verbatim}

\textcolor{blue}{**3. Summary of Data in Data Frame**}

The statistical summary and nature of the data can be obtained by
applying summary() function.

\begin{Shaded}
\begin{Highlighting}[]
\CommentTok{\# Create the data frame.}
\NormalTok{student.data }\OtherTok{\textless{}{-}} \FunctionTok{data.frame}\NormalTok{(}
   \AttributeTok{std\_id =} \FunctionTok{c}\NormalTok{ (}\DecValTok{1}\SpecialCharTok{:}\DecValTok{5}\NormalTok{), }
   \AttributeTok{std\_name =} \FunctionTok{c}\NormalTok{(}\StringTok{"Ali"}\NormalTok{,}\StringTok{"Abu"}\NormalTok{,}\StringTok{"Ahmad"}\NormalTok{,}\StringTok{"Siti"}\NormalTok{,}\StringTok{"Emma"}\NormalTok{),}
   \AttributeTok{pocketmoney =} \FunctionTok{c}\NormalTok{(}\FloatTok{623.3}\NormalTok{,}\FloatTok{515.2}\NormalTok{,}\FloatTok{611.0}\NormalTok{,}\FloatTok{729.0}\NormalTok{,}\FloatTok{843.25}\NormalTok{), }
   
   \AttributeTok{start\_date =} \FunctionTok{as.Date}\NormalTok{(}\FunctionTok{c}\NormalTok{(}\StringTok{"2012{-}01{-}01"}\NormalTok{, }\StringTok{"2013{-}09{-}23"}\NormalTok{, }\StringTok{"2014{-}11{-}15"}\NormalTok{, }\StringTok{"2014{-}05{-}11"}\NormalTok{,}
      \StringTok{"2015{-}03{-}27"}\NormalTok{)),}
  
   \AttributeTok{stringsAsFactors =} \ConstantTok{FALSE}
\NormalTok{)}
\CommentTok{\# Print the summary.}
\FunctionTok{print}\NormalTok{(}\FunctionTok{summary}\NormalTok{(student.data)) }
\end{Highlighting}
\end{Shaded}

\begin{verbatim}
##      std_id    std_name          pocketmoney      start_date        
##  Min.   :1   Length:5           Min.   :515.2   Min.   :2012-01-01  
##  1st Qu.:2   Class :character   1st Qu.:611.0   1st Qu.:2013-09-23  
##  Median :3   Mode  :character   Median :623.3   Median :2014-05-11  
##  Mean   :3                      Mean   :664.4   Mean   :2014-01-14  
##  3rd Qu.:4                      3rd Qu.:729.0   3rd Qu.:2014-11-15  
##  Max.   :5                      Max.   :843.2   Max.   :2015-03-27
\end{verbatim}

\textcolor{blue}{**4. Extract Data from Data Frame**}

Extract specific column from a data frame using column name.

\begin{Shaded}
\begin{Highlighting}[]
\CommentTok{\# Create the data frame.}
\NormalTok{student.data }\OtherTok{\textless{}{-}} \FunctionTok{data.frame}\NormalTok{(}
   \AttributeTok{std\_id =} \FunctionTok{c}\NormalTok{ (}\DecValTok{1}\SpecialCharTok{:}\DecValTok{5}\NormalTok{), }
   \AttributeTok{std\_name =} \FunctionTok{c}\NormalTok{(}\StringTok{"Ali"}\NormalTok{,}\StringTok{"Abu"}\NormalTok{,}\StringTok{"Ahmad"}\NormalTok{,}\StringTok{"Siti"}\NormalTok{,}\StringTok{"Emma"}\NormalTok{),}
   \AttributeTok{pocketmoney =} \FunctionTok{c}\NormalTok{(}\FloatTok{623.3}\NormalTok{,}\FloatTok{515.2}\NormalTok{,}\FloatTok{611.0}\NormalTok{,}\FloatTok{729.0}\NormalTok{,}\FloatTok{843.25}\NormalTok{), }
   
   \AttributeTok{start\_date =} \FunctionTok{as.Date}\NormalTok{(}\FunctionTok{c}\NormalTok{(}\StringTok{"2012{-}01{-}01"}\NormalTok{, }\StringTok{"2013{-}09{-}23"}\NormalTok{, }\StringTok{"2014{-}11{-}15"}\NormalTok{, }\StringTok{"2014{-}05{-}11"}\NormalTok{,}
      \StringTok{"2015{-}03{-}27"}\NormalTok{)),}

   \AttributeTok{stringsAsFactors =} \ConstantTok{FALSE}
\NormalTok{)}
\CommentTok{\# Extract Specific columns.}
\NormalTok{result }\OtherTok{\textless{}{-}} \FunctionTok{data.frame}\NormalTok{(student.data}\SpecialCharTok{$}\NormalTok{std\_name,student.data}\SpecialCharTok{$}\NormalTok{pocketmoney)}
\FunctionTok{print}\NormalTok{(result)}
\end{Highlighting}
\end{Shaded}

\begin{verbatim}
##   student.data.std_name student.data.pocketmoney
## 1                   Ali                   623.30
## 2                   Abu                   515.20
## 3                 Ahmad                   611.00
## 4                  Siti                   729.00
## 5                  Emma                   843.25
\end{verbatim}

Extract the first two rows and then all columns

\begin{Shaded}
\begin{Highlighting}[]
\CommentTok{\# Create the data frame.}
\NormalTok{student.data }\OtherTok{\textless{}{-}} \FunctionTok{data.frame}\NormalTok{(}
   \AttributeTok{std\_id =} \FunctionTok{c}\NormalTok{ (}\DecValTok{1}\SpecialCharTok{:}\DecValTok{5}\NormalTok{), }
   \AttributeTok{std\_name =} \FunctionTok{c}\NormalTok{(}\StringTok{"Ali"}\NormalTok{,}\StringTok{"Abu"}\NormalTok{,}\StringTok{"Ahmad"}\NormalTok{,}\StringTok{"Siti"}\NormalTok{,}\StringTok{"Emma"}\NormalTok{),}
   \AttributeTok{pocketmoney =} \FunctionTok{c}\NormalTok{(}\FloatTok{623.3}\NormalTok{,}\FloatTok{515.2}\NormalTok{,}\FloatTok{611.0}\NormalTok{,}\FloatTok{729.0}\NormalTok{,}\FloatTok{843.25}\NormalTok{), }
   
   \AttributeTok{start\_date =} \FunctionTok{as.Date}\NormalTok{(}\FunctionTok{c}\NormalTok{(}\StringTok{"2012{-}01{-}01"}\NormalTok{, }\StringTok{"2013{-}09{-}23"}\NormalTok{, }\StringTok{"2014{-}11{-}15"}\NormalTok{, }\StringTok{"2014{-}05{-}11"}\NormalTok{,}
      \StringTok{"2015{-}03{-}27"}\NormalTok{)),}

   \AttributeTok{stringsAsFactors =} \ConstantTok{FALSE}
\NormalTok{)}
\CommentTok{\# Extract first two rows.}
\NormalTok{result }\OtherTok{\textless{}{-}}\NormalTok{ student.data[}\DecValTok{1}\SpecialCharTok{:}\DecValTok{2}\NormalTok{,]}
\FunctionTok{print}\NormalTok{(result)}
\end{Highlighting}
\end{Shaded}

\begin{verbatim}
##   std_id std_name pocketmoney start_date
## 1      1      Ali       623.3 2012-01-01
## 2      2      Abu       515.2 2013-09-23
\end{verbatim}

Extract 3rd and 5th row with 2nd and 4th column

\begin{Shaded}
\begin{Highlighting}[]
\CommentTok{\# Create the data frame.}
\NormalTok{student.data }\OtherTok{\textless{}{-}} \FunctionTok{data.frame}\NormalTok{(}
   \AttributeTok{std\_id =} \FunctionTok{c}\NormalTok{ (}\DecValTok{1}\SpecialCharTok{:}\DecValTok{5}\NormalTok{), }
   \AttributeTok{std\_name =} \FunctionTok{c}\NormalTok{(}\StringTok{"Ali"}\NormalTok{,}\StringTok{"Abu"}\NormalTok{,}\StringTok{"Ahmad"}\NormalTok{,}\StringTok{"Siti"}\NormalTok{,}\StringTok{"Emma"}\NormalTok{),}
   \AttributeTok{pocketmoney =} \FunctionTok{c}\NormalTok{(}\FloatTok{623.3}\NormalTok{,}\FloatTok{515.2}\NormalTok{,}\FloatTok{611.0}\NormalTok{,}\FloatTok{729.0}\NormalTok{,}\FloatTok{843.25}\NormalTok{), }
   
   \AttributeTok{start\_date =} \FunctionTok{as.Date}\NormalTok{(}\FunctionTok{c}\NormalTok{(}\StringTok{"2012{-}01{-}01"}\NormalTok{, }\StringTok{"2013{-}09{-}23"}\NormalTok{, }\StringTok{"2014{-}11{-}15"}\NormalTok{, }\StringTok{"2014{-}05{-}11"}\NormalTok{,}
      \StringTok{"2015{-}03{-}27"}\NormalTok{)),}
 
   \AttributeTok{stringsAsFactors =} \ConstantTok{FALSE}
\NormalTok{)}

\CommentTok{\# Extract 3rd and 5th row with 2nd and 4th column.}
\NormalTok{result }\OtherTok{\textless{}{-}}\NormalTok{ student.data[}\FunctionTok{c}\NormalTok{(}\DecValTok{3}\NormalTok{,}\DecValTok{5}\NormalTok{),}\FunctionTok{c}\NormalTok{(}\DecValTok{2}\NormalTok{,}\DecValTok{4}\NormalTok{)]}
\FunctionTok{print}\NormalTok{(result)}
\end{Highlighting}
\end{Shaded}

\begin{verbatim}
##   std_name start_date
## 3    Ahmad 2014-11-15
## 5     Emma 2015-03-27
\end{verbatim}

\textcolor{blue}{**5. Expand Data Frame**}

A data frame can be expanded by adding columns and rows.

\textcolor{blue}{**Add Column**}

Just add the column vector using a new column name.

\begin{Shaded}
\begin{Highlighting}[]
\CommentTok{\# Create the data frame.}
\NormalTok{student.data }\OtherTok{\textless{}{-}} \FunctionTok{data.frame}\NormalTok{(}
   \AttributeTok{std\_id =} \FunctionTok{c}\NormalTok{ (}\DecValTok{1}\SpecialCharTok{:}\DecValTok{5}\NormalTok{), }
   \AttributeTok{std\_name =} \FunctionTok{c}\NormalTok{(}\StringTok{"Ali"}\NormalTok{,}\StringTok{"Abu"}\NormalTok{,}\StringTok{"Ahmad"}\NormalTok{,}\StringTok{"Siti"}\NormalTok{,}\StringTok{"Emma"}\NormalTok{),}
   \AttributeTok{pocketmoney =} \FunctionTok{c}\NormalTok{(}\FloatTok{623.3}\NormalTok{,}\FloatTok{515.2}\NormalTok{,}\FloatTok{611.0}\NormalTok{,}\FloatTok{729.0}\NormalTok{,}\FloatTok{843.25}\NormalTok{), }
   
   \AttributeTok{start\_date =} \FunctionTok{as.Date}\NormalTok{(}\FunctionTok{c}\NormalTok{(}\StringTok{"2012{-}01{-}01"}\NormalTok{, }\StringTok{"2013{-}09{-}23"}\NormalTok{, }\StringTok{"2014{-}11{-}15"}\NormalTok{, }\StringTok{"2014{-}05{-}11"}\NormalTok{,}
      \StringTok{"2015{-}03{-}27"}\NormalTok{)),}
 
   \AttributeTok{stringsAsFactors =} \ConstantTok{FALSE}
\NormalTok{)}

\CommentTok{\# Add the "program" column.}
\NormalTok{student.data}\SpecialCharTok{$}\NormalTok{program }\OtherTok{\textless{}{-}} \FunctionTok{c}\NormalTok{(}\StringTok{"CS"}\NormalTok{,}\StringTok{"BM"}\NormalTok{,}\StringTok{"Stat"}\NormalTok{,}\StringTok{"Banking"}\NormalTok{,}\StringTok{"AM"}\NormalTok{)}
\NormalTok{v }\OtherTok{\textless{}{-}}\NormalTok{ student.data}
\FunctionTok{print}\NormalTok{(v)}
\end{Highlighting}
\end{Shaded}

\begin{verbatim}
##   std_id std_name pocketmoney start_date program
## 1      1      Ali      623.30 2012-01-01      CS
## 2      2      Abu      515.20 2013-09-23      BM
## 3      3    Ahmad      611.00 2014-11-15    Stat
## 4      4     Siti      729.00 2014-05-11 Banking
## 5      5     Emma      843.25 2015-03-27      AM
\end{verbatim}

\textcolor{blue}{**Add Row**}

To \textbf{add more rows permanently} to an existing data frame, we need
to bring in the new rows in the \textbf{same structure} as the existing
data frame and use the \textbf{rbind()} function.

In the example below we create a data frame with new rows and merge it
with the existing data frame to create the final data frame.

\begin{Shaded}
\begin{Highlighting}[]
\CommentTok{\# Create the first data frame.}
\NormalTok{student.data }\OtherTok{\textless{}{-}} \FunctionTok{data.frame}\NormalTok{(}
   \AttributeTok{std\_id =} \FunctionTok{c}\NormalTok{ (}\DecValTok{1}\SpecialCharTok{:}\DecValTok{5}\NormalTok{), }
   \AttributeTok{std\_name =} \FunctionTok{c}\NormalTok{(}\StringTok{"Ali"}\NormalTok{,}\StringTok{"Abu"}\NormalTok{,}\StringTok{"Ahmad"}\NormalTok{,}\StringTok{"Siti"}\NormalTok{,}\StringTok{"Emma"}\NormalTok{),}
   \AttributeTok{pocketmoney =} \FunctionTok{c}\NormalTok{(}\FloatTok{623.3}\NormalTok{,}\FloatTok{515.2}\NormalTok{,}\FloatTok{611.0}\NormalTok{,}\FloatTok{729.0}\NormalTok{,}\FloatTok{843.25}\NormalTok{), }
   
   \AttributeTok{start\_date =} \FunctionTok{as.Date}\NormalTok{(}\FunctionTok{c}\NormalTok{(}\StringTok{"2012{-}01{-}01"}\NormalTok{, }\StringTok{"2013{-}09{-}23"}\NormalTok{, }\StringTok{"2014{-}11{-}15"}\NormalTok{, }\StringTok{"2014{-}05{-}11"}\NormalTok{,}
      \StringTok{"2015{-}03{-}27"}\NormalTok{)),}
   \AttributeTok{program =} \FunctionTok{c}\NormalTok{(}\StringTok{"CS"}\NormalTok{,}\StringTok{"BM"}\NormalTok{,}\StringTok{"Stat"}\NormalTok{,}\StringTok{"Banking"}\NormalTok{,}\StringTok{"AM"}\NormalTok{),}
   \AttributeTok{stringsAsFactors =} \ConstantTok{FALSE}
\NormalTok{)}

\CommentTok{\# Create the second data frame}
\NormalTok{student.newdata }\OtherTok{\textless{}{-}}  \FunctionTok{data.frame}\NormalTok{(}
   \AttributeTok{std\_id =} \FunctionTok{c}\NormalTok{ (}\DecValTok{6}\SpecialCharTok{:}\DecValTok{8}\NormalTok{), }
   \AttributeTok{std\_name =} \FunctionTok{c}\NormalTok{(}\StringTok{"Hairi"}\NormalTok{,}\StringTok{"Alimah"}\NormalTok{,}\StringTok{"Maria"}\NormalTok{),}
   \AttributeTok{pocketmoney =} \FunctionTok{c}\NormalTok{(}\FloatTok{578.0}\NormalTok{,}\FloatTok{722.5}\NormalTok{,}\FloatTok{632.8}\NormalTok{), }
   \AttributeTok{start\_date =} \FunctionTok{as.Date}\NormalTok{(}\FunctionTok{c}\NormalTok{(}\StringTok{"2013{-}05{-}21"}\NormalTok{,}\StringTok{"2013{-}07{-}30"}\NormalTok{,}\StringTok{"2014{-}06{-}17"}\NormalTok{)),}
   \AttributeTok{program =} \FunctionTok{c}\NormalTok{(}\StringTok{"CS"}\NormalTok{,}\StringTok{"Stat"}\NormalTok{,}\StringTok{"AM"}\NormalTok{),}
   \AttributeTok{stringsAsFactors =} \ConstantTok{FALSE}
\NormalTok{)}

\CommentTok{\# Bind the two data frames.}
\NormalTok{std.finaldata }\OtherTok{\textless{}{-}} \FunctionTok{rbind}\NormalTok{(student.data,student.newdata)}
\FunctionTok{print}\NormalTok{(std.finaldata)}
\end{Highlighting}
\end{Shaded}

\begin{verbatim}
##   std_id std_name pocketmoney start_date program
## 1      1      Ali      623.30 2012-01-01      CS
## 2      2      Abu      515.20 2013-09-23      BM
## 3      3    Ahmad      611.00 2014-11-15    Stat
## 4      4     Siti      729.00 2014-05-11 Banking
## 5      5     Emma      843.25 2015-03-27      AM
## 6      6    Hairi      578.00 2013-05-21      CS
## 7      7   Alimah      722.50 2013-07-30    Stat
## 8      8    Maria      632.80 2014-06-17      AM
\end{verbatim}

\hypertarget{factors}{%
\subsection{Factors}\label{factors}}

Factors are a unique data type in R used to represent categorical or
discrete data. They are particularly useful for working with data that
has distinct categories, such as nominal or ordinal variables. Factors
allow you to efficiently store and manage categorical data while
maintaining the underlying order or levels.

Factors are the data objects which are used to \textbf{categorize the
data} and store it as \textbf{levels}. They can store both strings and
integers. They are useful in the columns which have a limited number of
unique values. Like ``Male,''Female'' and True, False etc. They are
useful in data analysis for statistical modeling.

Factors are created using the \textbf{factor ()} function by taking a
vector as input.

\textbf{Example}

\begin{Shaded}
\begin{Highlighting}[]
\CommentTok{\# Create a vector as input.}
\NormalTok{data }\OtherTok{\textless{}{-}} \FunctionTok{c}\NormalTok{(}\StringTok{"East"}\NormalTok{,}\StringTok{"West"}\NormalTok{,}\StringTok{"East"}\NormalTok{,}\StringTok{"North"}\NormalTok{,}\StringTok{"North"}\NormalTok{,}\StringTok{"East"}\NormalTok{,}\StringTok{"West"}\NormalTok{,}\StringTok{"West"}\NormalTok{,}\StringTok{"West"}\NormalTok{,}\StringTok{"East"}\NormalTok{,}\StringTok{"North"}\NormalTok{)}

\FunctionTok{print}\NormalTok{(data)}
\end{Highlighting}
\end{Shaded}

\begin{verbatim}
##  [1] "East"  "West"  "East"  "North" "North" "East"  "West"  "West"  "West" 
## [10] "East"  "North"
\end{verbatim}

\begin{Shaded}
\begin{Highlighting}[]
\FunctionTok{print}\NormalTok{(}\FunctionTok{is.factor}\NormalTok{(data))}
\end{Highlighting}
\end{Shaded}

\begin{verbatim}
## [1] FALSE
\end{verbatim}

\begin{Shaded}
\begin{Highlighting}[]
\CommentTok{\# Apply the factor function.}
\NormalTok{factor\_data }\OtherTok{\textless{}{-}} \FunctionTok{factor}\NormalTok{(data)}

\FunctionTok{print}\NormalTok{(factor\_data)}
\end{Highlighting}
\end{Shaded}

\begin{verbatim}
##  [1] East  West  East  North North East  West  West  West  East  North
## Levels: East North West
\end{verbatim}

\begin{Shaded}
\begin{Highlighting}[]
\FunctionTok{print}\NormalTok{(}\FunctionTok{is.factor}\NormalTok{(factor\_data))}
\end{Highlighting}
\end{Shaded}

\begin{verbatim}
## [1] TRUE
\end{verbatim}

\textbf{Working with Factors:}

Factors have distinct properties that make them useful for data
analysis:

\begin{enumerate}
\def\labelenumi{\arabic{enumi}.}
\tightlist
\item
  \textbf{Levels}: Factors have predefined levels that represent the
  distinct categories in the data.
\item
  \textbf{Order}: Factors can be ordered, making them suitable for
  ordinal data.
\item
  \textbf{Labels}: Each level can have a human-readable label associated
  with it.
\end{enumerate}

\textcolor{blue}{**1. Factors in Data Frame**}

On creating any data frame with a column of text data, R treats the text
column as categorical data and creates factors on it.

\begin{Shaded}
\begin{Highlighting}[]
\CommentTok{\# Create the vectors for data frame.}
\NormalTok{height }\OtherTok{\textless{}{-}} \FunctionTok{c}\NormalTok{(}\DecValTok{132}\NormalTok{,}\DecValTok{151}\NormalTok{,}\DecValTok{162}\NormalTok{,}\DecValTok{139}\NormalTok{,}\DecValTok{166}\NormalTok{,}\DecValTok{147}\NormalTok{,}\DecValTok{122}\NormalTok{)}
\NormalTok{weight }\OtherTok{\textless{}{-}} \FunctionTok{c}\NormalTok{(}\DecValTok{48}\NormalTok{,}\DecValTok{49}\NormalTok{,}\DecValTok{66}\NormalTok{,}\DecValTok{53}\NormalTok{,}\DecValTok{67}\NormalTok{,}\DecValTok{52}\NormalTok{,}\DecValTok{40}\NormalTok{)}
\NormalTok{gender }\OtherTok{\textless{}{-}} \FunctionTok{c}\NormalTok{(}\StringTok{"male"}\NormalTok{,}\StringTok{"male"}\NormalTok{,}\StringTok{"female"}\NormalTok{,}\StringTok{"female"}\NormalTok{,}\StringTok{"male"}\NormalTok{,}\StringTok{"female"}\NormalTok{,}\StringTok{"male"}\NormalTok{)}

\CommentTok{\# Create the data frame.}
\NormalTok{input\_data }\OtherTok{\textless{}{-}} \FunctionTok{data.frame}\NormalTok{(height,weight,gender)}
\FunctionTok{print}\NormalTok{(input\_data)}
\end{Highlighting}
\end{Shaded}

\begin{verbatim}
##   height weight gender
## 1    132     48   male
## 2    151     49   male
## 3    162     66 female
## 4    139     53 female
## 5    166     67   male
## 6    147     52 female
## 7    122     40   male
\end{verbatim}

\begin{Shaded}
\begin{Highlighting}[]
\CommentTok{\# Test if the gender column is a factor.}
\FunctionTok{print}\NormalTok{(}\FunctionTok{is.factor}\NormalTok{(input\_data}\SpecialCharTok{$}\NormalTok{gender))}
\end{Highlighting}
\end{Shaded}

\begin{verbatim}
## [1] FALSE
\end{verbatim}

\begin{Shaded}
\begin{Highlighting}[]
\CommentTok{\# Print the gender column so see the levels.}
\FunctionTok{print}\NormalTok{(input\_data}\SpecialCharTok{$}\NormalTok{gender)}
\end{Highlighting}
\end{Shaded}

\begin{verbatim}
## [1] "male"   "male"   "female" "female" "male"   "female" "male"
\end{verbatim}

\textcolor{blue}{**2. Changing the Order of Levels**}

The order of the levels in a factor can be changed by applying the
factor function again with new order of the levels.

\begin{Shaded}
\begin{Highlighting}[]
\NormalTok{data }\OtherTok{\textless{}{-}} \FunctionTok{c}\NormalTok{(}\StringTok{"East"}\NormalTok{,}\StringTok{"West"}\NormalTok{,}\StringTok{"East"}\NormalTok{,}\StringTok{"North"}\NormalTok{,}\StringTok{"North"}\NormalTok{,}\StringTok{"East"}\NormalTok{,}\StringTok{"West"}\NormalTok{,}
   \StringTok{"West"}\NormalTok{,}\StringTok{"West"}\NormalTok{,}\StringTok{"East"}\NormalTok{,}\StringTok{"North"}\NormalTok{)}
\CommentTok{\# Create the factors}
\NormalTok{factor\_data }\OtherTok{\textless{}{-}} \FunctionTok{factor}\NormalTok{(data)}
\FunctionTok{print}\NormalTok{(factor\_data)}
\end{Highlighting}
\end{Shaded}

\begin{verbatim}
##  [1] East  West  East  North North East  West  West  West  East  North
## Levels: East North West
\end{verbatim}

\begin{Shaded}
\begin{Highlighting}[]
\CommentTok{\# Apply the factor function with required order of the level.}
\NormalTok{new\_order\_data }\OtherTok{\textless{}{-}} \FunctionTok{factor}\NormalTok{(factor\_data,}\AttributeTok{levels =} \FunctionTok{c}\NormalTok{(}\StringTok{"East"}\NormalTok{,}\StringTok{"West"}\NormalTok{,}\StringTok{"North"}\NormalTok{))}
\FunctionTok{print}\NormalTok{(new\_order\_data)}
\end{Highlighting}
\end{Shaded}

\begin{verbatim}
##  [1] East  West  East  North North East  West  West  West  East  North
## Levels: East West North
\end{verbatim}

\textcolor{blue}{**3. Generating Factor Levels**}

We can generate factor levels by using the gl() function. It takes two
integers as input which indicates how many levels and how many times
each level.

\textbf{Syntax}

\begin{Shaded}
\begin{Highlighting}[]
\FunctionTok{gl}\NormalTok{(n, k, labels)}
\end{Highlighting}
\end{Shaded}

Following is the description of the parameters used −

\begin{itemize}
\tightlist
\item
  \textbf{n} is a integer giving the number of levels.
\item
  \textbf{k} is a integer giving the number of replications.
\item
  \textbf{labels} is a vector of labels for the resulting factor levels.
\end{itemize}

Example

\begin{Shaded}
\begin{Highlighting}[]
\NormalTok{v }\OtherTok{\textless{}{-}} \FunctionTok{gl}\NormalTok{(}\DecValTok{3}\NormalTok{, }\DecValTok{4}\NormalTok{, }\AttributeTok{labels =} \FunctionTok{c}\NormalTok{(}\StringTok{"Tampa"}\NormalTok{, }\StringTok{"Seattle"}\NormalTok{,}\StringTok{"Boston"}\NormalTok{))}
\FunctionTok{print}\NormalTok{(v)}
\end{Highlighting}
\end{Shaded}

\begin{verbatim}
##  [1] Tampa   Tampa   Tampa   Tampa   Seattle Seattle Seattle Seattle Boston 
## [10] Boston  Boston  Boston 
## Levels: Tampa Seattle Boston
\end{verbatim}

\hypertarget{arithmetic-with-r}{%
\subsection{Arithmetic with R}\label{arithmetic-with-r}}

In its most basic form, R can be used as a simple calculator. Consider
the following arithmetic operators:

Addition: + Subtraction: - Multiplication: * Division: / Exponentiation:
\^{} Modulo: \%\% The last two might need some explaining:

The \^{} operator raises the number to its left to the power of the
number to its right: for example 3\^{}2 is 9. The modulo returns the
remainder of the division of the number to the left by the number on its
right, for example 5 modulo 3 or 5 \%\% 3 is 2.

\end{document}
