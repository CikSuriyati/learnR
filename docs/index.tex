% Options for packages loaded elsewhere
\PassOptionsToPackage{unicode}{hyperref}
\PassOptionsToPackage{hyphens}{url}
%
\documentclass[
]{article}
\usepackage{amsmath,amssymb}
<<<<<<< Updated upstream
\usepackage{lmodern}
=======
<<<<<<< HEAD
=======
\usepackage{lmodern}
>>>>>>> 71398288ce50cbb0a144a8dabbbc2dd9a6faf889
>>>>>>> Stashed changes
\usepackage{iftex}
\ifPDFTeX
  \usepackage[T1]{fontenc}
  \usepackage[utf8]{inputenc}
  \usepackage{textcomp} % provide euro and other symbols
\else % if luatex or xetex
<<<<<<< Updated upstream
=======
<<<<<<< HEAD
  \usepackage{unicode-math} % this also loads fontspec
  \defaultfontfeatures{Scale=MatchLowercase}
  \defaultfontfeatures[\rmfamily]{Ligatures=TeX,Scale=1}
\fi
\usepackage{lmodern}
\ifPDFTeX\else
  % xetex/luatex font selection
\fi
=======
>>>>>>> Stashed changes
  \usepackage{unicode-math}
  \defaultfontfeatures{Scale=MatchLowercase}
  \defaultfontfeatures[\rmfamily]{Ligatures=TeX,Scale=1}
\fi
<<<<<<< Updated upstream
=======
>>>>>>> 71398288ce50cbb0a144a8dabbbc2dd9a6faf889
>>>>>>> Stashed changes
% Use upquote if available, for straight quotes in verbatim environments
\IfFileExists{upquote.sty}{\usepackage{upquote}}{}
\IfFileExists{microtype.sty}{% use microtype if available
  \usepackage[]{microtype}
  \UseMicrotypeSet[protrusion]{basicmath} % disable protrusion for tt fonts
}{}
\makeatletter
\@ifundefined{KOMAClassName}{% if non-KOMA class
  \IfFileExists{parskip.sty}{%
    \usepackage{parskip}
  }{% else
    \setlength{\parindent}{0pt}
    \setlength{\parskip}{6pt plus 2pt minus 1pt}}
}{% if KOMA class
  \KOMAoptions{parskip=half}}
\makeatother
\usepackage{xcolor}
\usepackage[margin=1in]{geometry}
\usepackage{graphicx}
\makeatletter
\def\maxwidth{\ifdim\Gin@nat@width>\linewidth\linewidth\else\Gin@nat@width\fi}
\def\maxheight{\ifdim\Gin@nat@height>\textheight\textheight\else\Gin@nat@height\fi}
\makeatother
% Scale images if necessary, so that they will not overflow the page
% margins by default, and it is still possible to overwrite the defaults
% using explicit options in \includegraphics[width, height, ...]{}
\setkeys{Gin}{width=\maxwidth,height=\maxheight,keepaspectratio}
% Set default figure placement to htbp
\makeatletter
\def\fps@figure{htbp}
\makeatother
\setlength{\emergencystretch}{3em} % prevent overfull lines
\providecommand{\tightlist}{%
  \setlength{\itemsep}{0pt}\setlength{\parskip}{0pt}}
\setcounter{secnumdepth}{-\maxdimen} % remove section numbering
\usepackage{booktabs}
\usepackage{longtable}
\usepackage{array}
\usepackage{multirow}
\usepackage{wrapfig}
\usepackage{float}
\usepackage{colortbl}
\usepackage{pdflscape}
\usepackage{tabu}
\usepackage{threeparttable}
\usepackage{threeparttablex}
\usepackage[normalem]{ulem}
\usepackage{makecell}
\usepackage{xcolor}
\ifLuaTeX
  \usepackage{selnolig}  % disable illegal ligatures
\fi
\IfFileExists{bookmark.sty}{\usepackage{bookmark}}{\usepackage{hyperref}}
\IfFileExists{xurl.sty}{\usepackage{xurl}}{} % add URL line breaks if available
<<<<<<< Updated upstream
\urlstyle{same} % disable monospaced font for URLs
=======
<<<<<<< HEAD
\urlstyle{same}
=======
\urlstyle{same} % disable monospaced font for URLs
>>>>>>> 71398288ce50cbb0a144a8dabbbc2dd9a6faf889
>>>>>>> Stashed changes
\hypersetup{
  pdftitle={Learn R},
  hidelinks,
  pdfcreator={LaTeX via pandoc}}

\title{Learn R}
\author{}
\date{\vspace{-2.5em}}

\begin{document}
\maketitle

\hypertarget{learn-r}{%
\section{\texorpdfstring{\textbf{Learn R}}{Learn R}}\label{learn-r}}

R is a powerful, versatile and free statistical programming language,
which has become increasingly popular among industrial and academic data
analysis. These introduction course assume no previous coding
experiences in R or any other programming language.

This course will provide students with extensive skills of fast
computing on how to organize computations to access, transform, explore,
analyze data and produce results. The primary focus is on teaching the
concepts and vocabulary of statistical computing. The ultimate goal is
that the students would be able to work in an office, lab or as a
research assistant to do essential computations and that they would be
able to legitimately put computing skills on their resume.

The method of teaching and learning included lecture, lab work and
project which should be held in the statistics laboratory. The
assessments consist of project, quiz and test.

\hypertarget{assessments-breakdown}{%
\subsection{Assessments Breakdown}\label{assessments-breakdown}}

<<<<<<< Updated upstream
=======
<<<<<<< HEAD
=======
>>>>>>> Stashed changes
\begin{verbatim}
## Warning in !is.null(rmarkdown::metadata$output) && rmarkdown::metadata$output
## %in% : 'length(x) = 2 > 1' in coercion to 'logical(1)'
\end{verbatim}

<<<<<<< Updated upstream
=======
>>>>>>> 71398288ce50cbb0a144a8dabbbc2dd9a6faf889
>>>>>>> Stashed changes
\begin{table}[!h]

\caption{\label{tab:unnamed-chunk-1}Details of Continuous Assessment}
\centering
\begin{tabular}[t]{l|l|l}
\hline
Assessment Type & Assessment Description & % of Total Mark\\
\hline
Presentation & Face to face & 25%\\
\hline
Quiz & Online & 15%\\
\hline
Test & Lab Test & 30%\\
\hline
Written Report & Written Report & 30%\\
\hline
\end{tabular}
\end{table}

\hypertarget{scheme-of-work}{%
\subsubsection{Scheme of Work}\label{scheme-of-work}}

\includegraphics{images/logo.png}

\end{document}
